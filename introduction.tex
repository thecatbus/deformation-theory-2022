\section{Introduction}
Deformation theory is a ubiquitous part of modern algebra and geometry, and is
concerned with studying infinitesimal neighbourhoods of algebro-geometric
objects in families that they sit in. The goal of this exposition is to provide
an accessible introduction to various facets of the subject with sufficiently
verbose examples, without pursuing any direction in great depth. 

We begin by making the notion of a deformation problem precise in
\Cref{section-whats-a-deformation}. Broadly speaking, this will be a functor
that takes a base scheme and gives all deformation families of the object in
concern parametrised over that base. Studying all families over a fixed 
base is a worthy pursuit in its own right-- many theorems in algebraic geometry
are proven by showing an object in concern can be deformed into something
degenerate, which is hence easier to handle. One such result is Mori's theorem
on existence of rational curves in Fano varieties
\cite[a treatment of this result can be found in ][Section
10.1]{matsukiIntroductionMoriProgram2002}. In \Cref{section-affine-deformation}
we compute such deformation families over the base \(\Spec
k[\epsilon]/(\epsilon^2)\) as explicitly as possible. In particular, we show
that smooth affine varieties are characterised by their deformation families
over this base.

In \Cref{section-dmodules}, we show that the space of differential operators on a
complex variety is naturally a deformation of its cotangent bundle. This is the
starting point for a rich theory of \(D\)-modules, which allow us to talk about
systems of differential equations on algebraic varieties.

In addition to studying individual deformation families over a specified base, we
can also study properties of the deformation functor itself. For instance, the
(pro) representing object for a deformation functor, if it exists, is uniquely
defined and contains most of the information about the deformation problem at
hand. By associating an appropriate deformation problem to a system, modern
geometers are able to extract important invariants-- for instance,
Donovan-Wemyss's \emph{contraction algebras} pro-represent a deformation functor
associated to compoint du-Val singularities and capture most of the homological
information about the singularity and its resolutions
\cite[Section 5]{wemyssAspectsHomologicalMinimal}. 

In classical deformation theory (with commutative test objects), a theorem
of Schlessinger characterises when the deformation functor is (pro) representable
\cite[the result is discussed in][Section 16]{hartshorne_deformation_2010}. One
such example is the deformation functor associated to a point \(p\) in a variety
\(X=\Spec A\). As one might expect it is possible to recover the local geometry
of the variety near \(p\) entirely from the associated deformation theory, and
we show that completed local ring \(\widehat{A}_p\) pro-represents the
deformation functor. In \Cref{section-noncommutative-def}, we exhibit how
extending the deformation functor to non-commutative test objects makes a
similar result hold even for non-commutative algebras \(A\), despite the loss of
a geometric picture. 

By introducing Deligne's philosophy of associating deformation functors to
differentially graded lie algebras and \(A_\infty\)-algebras, we reprove Ed
Segal's result which uses deformation theory to derive a finite presentation for
the non-commutative completion \(\widehat{A}_p\). 

\subsubsection{Prerequisites and conventions} We assume familiarity with basic
algebraic geometry and commutative algebra--  at the level of a first course in
scheme theory \cite[such as][chapter II]{harts} should be more than sufficient.
Likewise, elementary notions in category theory and homological algebra are
freely used. Parts of the exposition mention symplectic geometry and theory of
differential equations, but not much will be lost if the reader has not seen
these.

Unless otherwise specified, we work with an algebraically closed field \(k\) of
characteristic zero (although neither of these assumptions might be strictly
necessary), and nothing will be lost by assuming \(k=\mathbb{C}\). A variety is
an integral separated scheme of finite type over \(k\).  We do not make any
distinction between a locally free sheaf and the total space of the
corresponding vector bundle.

\subsubsection{Acknowledgements} This exposition was written as a group project
for the Glasgow-Maxwell School (GLaMS), and we thank our supervisors Ivan
Cheltsov, Michael Wemyss, and David Jordan for encouraging us to pursue this
topic. The second author also thanks Matt Booth for helpful conversations. 

We make no claim to originality, and while we have made an attempt to provide
references throughout the text, we mention here the most helpful resources
used-- \cite{szendroi_unbearable_1999, hartshorne_deformation_2010,
belmans_hochschild_2018} for classical deformation
theory of algebras and schemes, \cite{bellamy2016noncommutative, hotta2007d,
ginzburg1998lectures, okitaniIntroductionDmodules2022} for D-modules,
\cite{august_differentially_2015} for differentially graded (lie) algebras and
their relation to deformation theory, and \cite{eriksen_noncommutative_2017,
eriksenComputingNoncommutativeDeformations2014, segal__2008} for non-commutative
deformation theory.
