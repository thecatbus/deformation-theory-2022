\section{What is a deformation?}\label{section-whats-a-deformation}

We explore how deformation problems naturally arise when considering
algebro-geometric objects.

\begin{example}
    \label{example-double-point-1}
    Consider the affine scheme \(X_0=\Spec k[x]/(x^2)\), the double point. This
    can be seen as `limiting case' of the scheme \(X_t = \Spec k[x]/(x^2-t)\)
    for \(t\in k\): if \(t\neq 0\) then \(X_t\) is the reduced scheme with two
    points. We will eventually show this is the only non-trivial way to deform
    \(X_0\) as an abstract scheme, and this should agree with our intuition that
    the only way to deform a double point is to separate its components.

    On the other hand, consider \(X_0\) with its embedding into
    \(\mathbb{A}^1\) as the double point at \(0\). Deforming \(X_0\) in the
    ambient space, we have a second family of embedded subschemes \(X'_t\) given
    by \(\mathbb{V}((x-t)^2) \subset \Spec k[x]\). This corresponds to
    deforming \(X_0\) by translating it, without changing its intrinsic
    geometry. This deformation is genuinely distinct from both the trivial
    deformation (i.e.\ the constant family) and a deformation which separates
    the points.

    Lastly we can instead consider the \(k[x]\)-module \(M_0 = k[x]/(x^2)\), as
    a sheaf over \(\mathbb{A}^1\). This is simply the structure sheaf of
    \(\mathscr{O}_{X_0}\) pushed forward along the inclusion, and the family
    \(\mathscr{O}_{X'_t}\) can be considered a deformation of \(M_0\)
    as a module. Another family is given by \(M_t = k^{\oplus 2}\) where \(x\)
    acts as \((a,b) \mapsto (bt, a)\). Observe that this cannot be realised as
    the structure sheaf of an embedded deformation of \(X_0\), since
    \(M_t\) cannot be expressed as a quotient of \(k[x]\) for \(t\neq 0\).
\end{example}

Thus any structure we can associate multiple deformation problems-- a choice
which is made based on the broader context of what we are studying. To establish
a common framework for such problems, we begin by making the three notions of
deformations seen in \Cref{example-double-point-1} precise in a way that lends
itself to functorial formulation.

\subsection{Various deformation problems}
\label{subsec-various-deformation-problems}

We begin by outlining reasonable candidates for what one might call a
`deformation problem'-- given a structure (a scheme, a sheaf, a module etc)
\(X\), we wish to find a `nice family' containing \(X\), and if possible
classify such families up to isomorphism. The way to make this notion of a `nice
family' precise usually depends on the specific structure we wish to deform and
the aspects which we wish to preserve.

\subsubsection{Abstract deformations of schemes}
Fix a \(k\)-scheme \(T\). The category of \(T\)-schemes has objects given by diagrams
\(X\xrightarrow{\;\pi\;}T\) in \(\Sch/k\). We will often abuse notation and
write \(X_T\) for this object. For each scheme-theoretic point \(t\rightarrow
T\), the fiber 
\[X_t := \pi^{-1}(t) = t\times_T X_T \] 
is a scheme over \(t\), allowing us to think of \(X_T\) as a family of schemes
parametrised by \(T\). However for this interpretation to be sensible, we desire
the fibers to be ``similar''-- this is ensured by requiring the morphism \(\pi\)
to be flat. Accordingly, define a \textit{\(T\)-family of schemes} is a flat
morphism \(X_T\xrightarrow{\;\pi\;}T\) of \(k\)-schemes.

An \textit{augmentation} of the \(k\)-scheme \(T\) is the choice of a
\(k\)-point \(\mathbold{0} \hookrightarrow T\) (called the \textit{special
point}). Often there will be a natural choice of augmentation-- for instance if
\(T\) is the spectrum of a local \(k\)-algebra, there is a unique \(k\)-point
corresponding to the maximal ideal. 

If \(T\) is augmented and \(X_T\) is a \(T\)-scheme, the fiber \(X_\mathbold{0}\)
over the special point is called the \textit{special fiber}.

\begin{definition}
    \label{def-defthy-abstractscheme}
    For \(T,X\in \Sch/k\) such that \(T\) is augmented, a
    \textit{\(T\)-deformation} of \(X\) is a \(T\)-family
    \(X_T\rightarrow T\) and an isomorphism \(\varphi: X\rightarrow
    X_\mathbold{0}\) identifying the special fiber with \(X\).
\end{definition}

Often we will simply write \(X_T\) for such a deformation, leaving the
morphism to \(T\) and the isomorphism \(\varphi\) implicit. 

We say the \(T\)-deformations \(X_T\) and \(X'_T\) are isomorphic if there is
an isomorphism of \(T\)-schemes \(X_T\xrightarrow{\sim} X'_T\) which respects
the identifications of the special fibers with \(X\).

\subsubsection{Embedded deformations}
While we deformed schemes abstractly above, we can instead deform them inside an
ambient space. For a \(k\)-scheme \(X\), a \textit{\(T\)-family of closed
subschemes in \(X\)} is given by commutative diagram  
\[\begin{tikzcd}
    Z_T \arrow[r,"\iota",hook] \arrow[rd,"\pi"'] 
    & X\times T \arrow[d,"\text{pr}_2"] \\ 
    & T
\end{tikzcd}\] 
such that \(\iota\) is a closed immersion and \(\pi\) is flat. In particular,
\(Z_T\) is a \(T\)-family such that for any point \(t\hookrightarrow T\) the
fiber \(Z_t\) is naturally a closed subscheme of \(X\times t\) via the
inclusion \(\iota\). 

If \(T\) is augmented, note that the special fiber is a closed subscheme of \(X\).

\begin{definition} \label{def-embdef}
    Let \(T\) be an augmented \(k\)-scheme, and \(Z\xhookrightarrow{\;i\;}X\) a
    closed immersion in \(\Sch/k\). A \textit{\(T\)-deformation of \(Z\) in
    \(X\)} is a \(T\)-family \(Z_T\xhookrightarrow{\;\iota\;} X\times T\) 
    of closed subschemes in \(X\) and an isomorphism \(\varphi: Z\rightarrow
    Z_\mathbold{0}\) such that \(\iota|_{Z_\mathbold{0}}\circ \varphi = i\) .
\end{definition}

\subsubsection{Deformations of sheaves} Embedded deformations are nice
because we have the ambient space to work with-- we are looking for quotients
\(\mathscr{O}_{Z_T}\) of \(\mathscr{O}_{X\times T}\) which are flat over \(T\),
and give \(\mathscr{O}_Z\) upon restriction to \(X\). More generally, we can
consider deformations of coherent sheaves over an augmented \(k\)-scheme \(T\).

\begin{definition} \label{def-defthy-of-sheaf}
    If \(\mathcal{F}\) is a coherent sheaf on a \(k\)-scheme \(X\), then a
    \textit{\(T\)-deformation} of \(\mathcal{F}\) is a coherent sheaf
    \(\mathcal{F}_T\) on \(X\times T\), flat over \(T\), with an isomorphism
    \(\varphi: \mathcal{F}\rightarrow\mathcal{F}_T|_{X\times \mathbold{0}}\).
\end{definition}

Observe that embedded \(T\)-deformations of \(Z\hookrightarrow X\) are
precisely those deformations of the sheaf \(\mathscr{O}_Z\) on \(X\) which can
be expressed as quotients of the sheaf \(\mathscr{O}_{X\times T}\).
\Cref{example-double-point-1} exhibits that in most cases there are more
deformations of the sheaf \(\mathscr{O}_Z\) than there are deformations of the
associated embedded subscheme. 

\subsection{Functoriality in the parameter space}

A morphism of augmented \(k\)-schemes is a \(k\)-scheme morphism \(T'\rightarrow
T\) such that the natural diagram 
\[\begin{tikzcd}
    T' \arrow[r] & T \\
    \Spec k \arrow[u] \arrow[r, Rightarrow, no head] & \Spec k \arrow[u]
\end{tikzcd}\]
commutes, where the vertical arrows are the augmentation morphisms. This allows
us to define a category \(\Sch^1/k\) (the category of \(k\)-schemes with one
marked \(k\)-point), whose objects are augmented \(k\)-schemes and morphisms are
of the kind described above.

If \(T'\rightarrow T\) is a morphism of augmented \(k\)-schemes, and
\(X_T\rightarrow T\) is a \(T\)-deformation of \(X\in \Sch/k\), then the
pullback
\[ \begin{tikzcd}
    X_{T'} \arrow[r] \arrow[d] \arrow[dr, phantom, "\square"] & X_T
    \arrow[d] \\ 
    T' \arrow[r] & T 
\end{tikzcd} \]
gives a \(T'\)-deformation \(X_{T'}\) of \(X\). Writing \(\Def_X(T)\) for the
set of abstract \(T\)-deformations of \(X\) up to isomorphism, we see that we
in fact have a functor \((\Sch^1/k)^\text{op}\rightarrow \text{Set}.\)

The pullback construction works analogously for embedded deformations and
deformations of sheaves.

\subsubsection{Size of the parameter space} 
Global information carried by deformation families is often largely irrelevant,
and we mostly care about deformations in an (analytic or infinitesimal)
neighbourhood of the special point. Thus working with parameter spaces in
\(\Sch^1_k\) is not optimal, and we work instead with spectra of complete
(Artinian in the infinitesimal case) local rings.

Write \(\text{cArt}^1_k\) for the category of commutative Artinian local
\(k\)-algebras with residue field \(k\), and \(\widehat{\text{cArt}}{}^1_k\) for
the category of complete local \(k\)-algebras with residue field \(k\). It is
immediate that 
\[\text{cArt}^1_k \quad \subset \quad \widehat{\text{cArt}}{}^1_k \quad \subset
\quad (\Sch^1_k)^\text{op}\]
as full subcategories, where the last inclusion is via \(\Spec\). Then all the
deformation functors defined in \Cref{subsec-various-deformation-problems} can
be restricted to functors on \(\text{cArt}^1_k\) instead.

For \(R\in \widehat{\text{cArt}}{}^1_k\), write \(\mathbold{0}_R:
R\twoheadrightarrow k\) for the augmentation morphism and \(\mathfrak{m}_R\) for
the maximal ideal.

\subsection{ Deformation theory of a point}
\label{subsec-defthy-of-a-point}
As an extended example of the theory developed above, we study the deformation
theory of a closed point in a scheme. Given a \(k\)-scheme \(X\) with a
\(k\)-point \(x\), our intuition suggests that the infinitesimal deformation
theory of structures associated to \(x\) should only depend on an analytic
neighbourhood of \(x\) in \(X\). We make this notion precise by showing that the
functor \(\Def_{x\hookrightarrow X}\) is naturally isomorphic to
\(\text{Hom}(\widehat{\mathscr{O}}_{X,x}, -)\). 

First, observe that we can focus on affine schemes in light of the following
lemma.

\begin{lemma}\label{lemma-defthy-is-affine}
    If \(U\subset X\) is an open subscheme containing \(x\), then for any \(R\in
    \text{cArt}{}^1_k\) the natural map 
    \[\Def_{\mathscr{O}_{X,x}}(R) \rightarrow \Def_{\mathscr{O}_{U,x}}(R)\]
    (given by pulling back families) is a bijection. Thus the deformation theory
    of the skyscraper sheaf \(\mathscr{O}_{X,x}\) only depends on a
    neighbourhood of \(x\).
    \begin{proof}
        Observe that the scheme \(X\times \Spec R\) has the same
        underlying topological space as \(X\), but with structure sheaf
        \(\mathscr{O}_X\otimes R\). An \(R\)-deformation of
        \(\mathscr{O}_{X,x}\) is given by a \(\mathscr{O}_X\otimes
        R\)-module \(\mathcal{F}\) such that each stalk \(\mathcal{F}_p\) is a
        flat \(R\)-module satisfying \(\mathcal{F}_p\otimes_R k \cong
        (\mathscr{O}_{X,x})_p\). But flat modules over Artinian local rings are
        free, so we must have that \(\mathcal{F}\) is a skyscraper sheaf at
        \(x\). Thus the restriction map 
        \begin{align*}
            \Def_{\mathscr{O}_{X,x}}(R)&\rightarrow \Def_{\mathscr{O}_{U,x}}(R)
            \\ 
            \mathcal{F}&\mapsto \mathcal{F}|_U
        \end{align*}
        admits an inverse which is extension by zero on \(X\setminus U\).
    \end{proof}
\end{lemma}

This allows us to assume \(X\) is an affine scheme \(\Spec A\) for the rest of
this discussion, and the point \(x\) is given by a surjection
\(A\twoheadrightarrow k\) which we shall also call \(x\). Write
\(\mathfrak{m}_x\) for the kernel of this map. 

From the above discussion, any \(R\)-deformation of \(\mathscr{O}_{X,x}\) is
given by an \(A\)-module structure on the free \(R\)-module of rank \(1\) such
that the diagram below commutes.  
\begin{equation}\label{diag-deformation-of-point} \tag{\(\ast\)}
    \begin{tikzcd} 
        k & R \arrow[l,"\mathbold{0}_R"', twoheadrightarrow] \\ A\arrow[r]
        \arrow[u, "x", twoheadrightarrow] & A\otimes R \arrow[u] 
    \end{tikzcd} 
\end{equation} 
Here \(A\otimes R \rightarrow R\) is the map of \(A,R\)-bimodules given by
\(a\otimes r \mapsto ar\). Observe that the image of \(A\otimes R \rightarrow
R\) (which is an \(R\)-submodule of \(R\), i.e.\ an ideal) is not contained in
\(\mathfrak{m}_R\). Thus the map \(A\otimes R \rightarrow R\) must be
surjective, and the kernel is an ideal so we have a closed immersion \(\Spec R
\hookrightarrow X\times \Spec R\) which sends the special point to \(x\). This
is precisely an embedded deformation of \(x\hookrightarrow X\), and it is
immediate that all embedded deformations must arise in this way. Thus the
deformation theory of a point in \(X\) coincides with the deformation theory of
its skyscraper sheaf.

We will now show that this deformation theory only depends on the completion
\(\widehat{A}_{\mathfrak{m}_x}\).

\begin{proposition}\label{prop-defthy-of-a-point}
    Let \(X\) be a \(k\)-scheme with a \(k\)-point \(x\). Then
    for any \(R\in \text{cArt}{}^1_k\), the following sets are natural
    bijection:
    \begin{enumerate}[(i)] 
        \item the set of embedded \(R\)-deformations of \(x\) in \(X\),
        \item the set of \(R\)-deformations of the skyscraper sheaf
            \(\mathscr{O}_{X,x}\),
        \item the set of morphisms \(\Spec R\rightarrow X\) that send the
            special point \(\mathbold{0}_R\hookrightarrow \Spec R\) to \(x\),
            and 
        \item \(\Hom(\widehat{\mathscr{O}}_{X,x},R)\), the set of local
            \(k\)-algebra homomorphisms from the completion of the local ring of
            \(\mathscr{O}_{X,x}\) to \(R\).
    \end{enumerate}
    \begin{proof}
        We work in the affine case. The bijection
        \(\text{(i)}\leftrightarrow\text{(ii)}\) was shown in the discussion
        above. 

        Given a \(R\)-deformation of \(\mathscr{O}_{X,x}\), we can again
        construct the diagram (\ref{diag-deformation-of-point}). The composite
        map \(A\rightarrow R\) gives a morphism \(\Spec R \rightarrow X\) which
        sends the special point to \(x\). On the other hand, given a map
        \(A\rightarrow R\) such that the composition \(A\rightarrow R
        \twoheadrightarrow k\) is \(x\), we can factor the map \(A\rightarrow
        R\) as \(A\hookrightarrow A\otimes R \rightarrow R\) to get the
        commutative diagram of the form above, giving an \(R\)-deformation of
        \(\mathscr{O}_{X,x}\). This is the required bijection
        \(\text{(ii)}\leftrightarrow \text{(iii)}\).

        Lastly, if \(A\rightarrow R\) is a map such that the composition
        \(A\rightarrow R \twoheadrightarrow k\) is \(x\), then any element of
        \(A\setminus \mathfrak{m}_A\) must land in \(R\setminus \mathfrak{m}_R\)
        and thus has an invertible image. This gives an induced map
        \(A_{\mathfrak{m}_x}\rightarrow R\). Since \(R\) is artinian, the map
        factors through \(A_{\mathfrak{m}_x}/\mathfrak{m}_x^n\rightarrow R\) for
        some \(n\), giving an induced map
        \(\widehat{A}_{\mathfrak{m}_x}\rightarrow R\). Conversely, a map
        \(\widehat{A}_{\mathfrak{m}_x}\rightarrow R\) must factor through 
        \(A_{\mathfrak{m}_x}/\mathfrak{m}_x^n \cong
        \widehat{A}_{\mathfrak{m}_x}/\mathfrak{m}_x^n\rightarrow R\) for some
        \(n\). This gives the required map \(A\rightarrow R\), and the bijection
        \(\text{(iii)}\leftrightarrow \text{(iv)}\).
    \end{proof}
\end{proposition}

\subsubsection{Prorepresentability} In most cases the completed local ring
\(\widehat{\mathscr{O}}_{X,x}\) is not Artinian, so the functor
\(\Def_{x\hookrightarrow X}\) is not representable.  However, we do have
\(\widehat{\mathscr{O}}_{X,x}\in \widehat{\text{cArt}}{}^1_k \), so that 
\[\Def_{x\hookrightarrow X}(R) = \lim_{\longleftarrow}
\Hom(\widehat{\mathscr{O}}_{X,x}/\mathfrak{m}_x^n,
R),\]
where \(\mathfrak{m}_x\subset \widehat{\mathscr{O}}_{X,x}\) is the maximal
ideal. The inverse system \((\widehat{\mathscr{O}}_{X,x}/\mathfrak{m}_x^n)_{n\in
\mathbb{N}}\) does lie in \(\text{cArt}^1_k\), and we say the deformation
functor is \textit{prorepresented} by \(\widehat{\mathscr{O}}_{X,x}\). 

Given a category \(\mathbf{C}\) with procategory \(\widehat{\mathbf{C}}\) (i.e.\
\(\mathbf{C}\) is a full subcategory of \(\mathbf{\widehat{C}}\) and every object of
\(\widehat{\mathbf{C}}\) is the limit of some inverse system in \(\mathbf{C}\)),
we can define the notion of prorepresentability of functors as above-- say a
functor \(F: \mathbf{C}\rightarrow \text{Set}\) is prorepresented by
\(X=\underset{\leftarrow}{\lim}X_n\) if there are natural bijections \[F(Y) \cong
\lim_{\longleftarrow}\Hom(X_n,Y).\]  
Using Yoneda's lemma, we can show that the prorepresenting object is unique if it
exists. 

\subsection{Deformations as local moduli}
Recall a \textit{moduli functor} is a functor \((\Sch_k)^\text{op}\rightarrow
\text{Set}\) which assigns to each \(k\)-scheme \(T\) the set of `families of
objects over \(T\) up to equivalence'. Every \(k\)-scheme \(X\) determines a
moduli functor \(\text{Mor}(-,X)\). If a moduli functor \(F\) is naturally
isomorphic to \(\text{Mor}(-,X)\) for some \(X\in \Sch_k\), we say \(X\) is the
\textit{(fine) moduli space} associated to \(F\). By Yoneda's lemma, the moduli
space is unique up to unique isomorphism whenever it exists. 

Note that moduli functors are completely determined by their restriction to the
full subcategory of affine schemes \((\text{AffSch}_k)^\text{op}\subset
(\Sch_k)^\text{op}\). But this is equivalent to the category of commutative
\(k\)-algebras, so we can think of moduli functors as functors 
\(\text{cAlg}_k\rightarrow \text{Set}.\)  

Given a reasonable moduli functor \(F: \text{cAlg}_k \rightarrow \text{Set}\)
and an object \(Z\in F(k)\), there is an associated deformation problem given by 
\begin{align*}
    \Def_Z: \text{cArt}^1_k &\rightarrow \text{Set} \\ 
    R &\mapsto \{Z_R\in F(R) \;|\; F(\mathbold{0}_R)(Z_R)=Z \}.
\end{align*}
If the functor \(F\) is represented by the moduli space \(X\), then by
\Cref{prop-defthy-of-a-point} we see that the deformation functor \(\Def_X\)
gives precisely the embedded deformations of the \(k\)-point in \(X\)
corresponding to \(Z\). Then, for example, we can read off various local
properties of \(X\) near this point (the dimension, reducedness, the completed
local ring etc.) by simply examining the deformation functor. Indeed this is
one of the motivations for studying deformation theory-- even if a certain moduli
space exists as scheme it is typically very hard to construct, but deformation
theory allows us to study its local properties nonetheless. In cases where a
moduli space doesn't exist, deformation theory provides some insight into the
reasons why.

\begin{example} 
    The Hilbert functor of closed subschemes in a \(k\)-scheme \(X\) is given on
    objects of \(\Sch/k\) by 
    \[\text{Hilb}_{X/k}(T) = \{\text{\(T\)-families of closed subschemes in
    \(X\)}\},\] 
    and acts on morphisms by pulling back families. By a theorem of
    Grothendieck, this functor is representable and the moduli space
    \(\text{\underline{Hilb}}_{X/k}\) is a projective \(k\)-scheme. The
    \(k\)-points of \(\text{\underline{Hilb}}_{X/k}\) correspond to closed
    subschemes of \(X\).  Fix such a \(k\)-point \(p\), corresponding to
    \(Z\hookrightarrow X\). The associated deformation functor is given by 
    \[\Def_{Z\hookrightarrow X}(R)=\{\text{\(R\)-deformations of \(Z\) in
    \(X\)}\}.\]
\end{example}

% Any \((R, \mathfrak{m}_R)\in \widehat{\text{cArt}}{}^1/k\) can be seen as the
% limit of an inverse system in \(\text{cArt}^1/k\), as
% \[R\cong \lim_{\longleftarrow} R/\mathfrak{m}_R^n.\]
% Thus a functor \(F: \text{cArt}^1/k \rightarrow \text{Set}\) can be extended to
% \(\widehat{\text{cArt}}{}^1/k\) via 
% \[\hat{F}(R) = \lim_{\longleftarrow} F(R/\mathfrak{m}_R^n).\]
% We say \(\hat{F}(R)\) is the set of \textit{formal families} of \(F\) over
% \(R\): an element of \(\hat{F}(R)\) is given by a sequence \((\xi_1,\xi_2,...)\)
% such that \(\xi_n\in F(R/\mathfrak{m}_R^n)\), and \(\xi_n \mapsto \xi_{n+1}\)
% under the map induced by \(R/\mathfrak{m}_R^n \twoheadrightarrow
% R/\mathfrak{m}_R^{n+1}\). Note that for a functor
% \(F:\widehat{\text{cArt}}{}^1/k\rightarrow \text{Set}\) and a complete
% \(k\)-algebra \(R\) there is a natural map \(F(R)\rightarrow \hat{F}(R)\), but
% this generally is neither injective nor surjective.
% 
% \subsection{Deformation functors}
% 
% We can try to understand the various notions of deformations on equal footing,
% as functors \(\text{cArt}^1/k \rightarrow \text{Set}\). We now outline
% properties which characterise such deformation functors. 
% 
% \textbf{Locality.} Say a functor \(F: \text{cArt}^1/k\rightarrow \text{Set}\) is
% \textit{local} if it satisfies
% \[\label{H0} \tag{H\(0\)} F(k)\text{ is the singleton set}.\] 
% For deformation problems, the unique element in this set shall be the structure
% we wish to deform. 
% 
% \textbf{Formal smoothness.} A natural transformation \(F\rightarrow G\)
% of local functors is \textit{smooth} if for every surjection
% \(S\twoheadrightarrow R\) in \(\text{cArt}^1/k\), the natural map 
% \[F(S) \rightarrow G(S)\times_{G(R)}F(R)\]
% is surjective. In words, we require that for any \(\xi\in G(S)\), if its image
% \(\bar\xi \in G(R)\) has a lift under the map \(F(R)\rightarrow G(R)\), then
% \(\xi\) has a compatible lift under \(F(S)\rightarrow G(S)\). Thus this property
% should be considered parallel to the infinitesimal lifting property of smooth
% morphisms of schemes, which is also called formal smoothness.
% 
% 
% 
% \section{Deformation functors}
%  To motivate the definition, 
% \textbf{Notation.} Write \(\text{cArt}^1_k\) for the full subcategory of
% Artinian local algebras in \(\text{cAlg}/k\). The object \(k\) is both initial
% and final in this category, so any \(R\in \text{cArt}^1_k\) is canonically
% augmented-- we write \(\mathbold{0}_R:R\twoheadrightarrow k\) for the augmentation
% morphism, and \(\bar{R}:= \ker(\mathbold{0}_R)\) for the maximal ideal.
% 
% 
% Thus deformation theory provides a lens to study local properties of moduli
% spaces. Accordingly, we shall define various properties desirable in a
% deformation functor.
% 
% 
% \textbf{Gluing over small extensions.}
% 
% \textbf{Dimension.} Write \(D:=k(\epsilon)/(\epsilon^2)\) for the ring of dual
% numbers. If \(X\) is a \(k\)-scheme, then a morphism \(\Spec D \rightarrow X\)
% is given by a \(k\)-point \(x\in X\) and a tangent vector \(v\in T_xX\).
% 
% \textbf{Smoothness.}
% 



