\section{Deformations of affine schemes}
\label{section-affine-deformation}

In this section we study the deformation theory of affine \(k\)-schemes
(equivalently \(k\)-algebras) as explicitly as possible-- first in an
embedded setting, and then as abstract schemes.  

\subsubsection{First-order deformations} If \(\Def: \text{cArt}^1_k \rightarrow
\text{Set}\) is any deformation functor, we say its \textit{tangent space} is
given by the set \(\Def(k[\epsilon]/(\epsilon^2)\). Here
\(k[\epsilon]/(\epsilon^2)\) is the ring of \textit{dual numbers}. We say each
element of the tangent space is a \textit{first-order} deformation. Any \(R\in
\text{cArt}^1_k\) admits surjections onto \(k[\epsilon]/(\epsilon^2)\), so every
\(R\)-deformation has to be the lift of some first-order deformation. Thus
studying first-order deformations is a step towards understanding deformations
over other rings, and that is the aim of this section.

All algebras considered in this section will be commutative. Abusing notation,
we will write \(k[\epsilon]\) for the ring of dual numbers and
\(R[\epsilon]=R\otimes_k k[\epsilon]\) for any \(k\)-algebra \(R\), leaving it
implicit that \(\epsilon^2=0\). Moreover, for this section all deformations will
be over \(k[\epsilon]\) so we will omit the term `first-order', referring to
them simply as \textit{deformations}. 

\begin{remark}
    The terminology `tangent space' comes from the following observation-- a
    \(k\)-scheme morphism \(\Spec k[\epsilon]\rightarrow X\)
    (i.e.\ a \(k[\epsilon]\)-point in \(X\)) is given by a
    \(k\)-point \(x\in X\) and a vector \(\theta \in T_x(X)\) in the tangent
    space of \(X\) at \(x\). In particular if \(X\) is the moduli space for a
    moduli functor \(F\), and \(\Def_Z\) is the deformation functor associated
    to \(Z\in F(k)\), then \(\Def_Z(k[\epsilon])\) is precisely the
    tangent space to \(X\) at the point corresponding to \(Z\).
\end{remark}

We have the following characterisation of flatness for modules over
$k[\epsilon]$, which we will use repeatedly:
\begin{lemma} 
    \label{flatness} 
    Let $M$ be a $k[\epsilon]$-module. Then $M$ is flat if and only if tensoring
    the exact sequence 
    \[0\rightarrow k \xrightarrow{\;\epsilon\;} k[\epsilon] \longrightarrow k
    \rightarrow 0\]
    with \(M\) preserves exactness, where the second non-zero map is the
    quotient by $(\epsilon)$.  
    \begin{proof} 
        A well-known criterion for flatness \cite[see for example][proposition
        6.1]{eisen} is that $M$ is flat if and only if $\text{Tor}_1^R(M,R/I)=0$
        for all finitely generated ideals $I\subset R$, for any ring $R$. When
        $R=k[\epsilon]$, the only non-trivial ideal is $I=(\epsilon)$, and
        $R/I\cong k$. Then since 
        \[0 \to M\otimes_{k[\epsilon]}k\to M\to
        M\otimes_{k[\epsilon]}k\to0 \] 
        is exact by hypothesis, we have $\text{Tor}_1^R(M,k)=0$ and $M$ is flat.  
    \end{proof}
\end{lemma}

\subsection{Embedded deformations}
Fix an ambient scheme \(X=\Spec S\), then a closed subscheme \(Y=\Spec R\) is
determined by a surjection \(S\twoheadrightarrow R\) of \(k\)-algebras. In this
setting, we can rephrase \Cref{def-embdef} as follows.

\begin{definition} 
    Suppose we have a surjection $S\to R$ of $k$-algebras with kernel $I$. An
    \textit{embedded (first-order) deformation} of $R$ in $S$ is an ideal
    $\t{I}\subset S[\epsilon]$ such that $\wt{R}=S[\epsilon]/\t{I}$ is flat over
    $k[\epsilon]$, and such that the map $S[\epsilon]\to S$ carries $\t{I}$ onto
    $I$. 
\end{definition}

In fact over \(k[\epsilon]\), embedded deformations of \(S\twoheadrightarrow R\)
are equivalent to deformations of the ideal \(I\) in a sense that we make
precise below, and we will use the two notions interchangeably. 

\begin{lemma}
    The following are equivalent.
    \begin{enumerate}
        \item \(\t{I}\subset S[\epsilon]\) is an ideal that is flat over
            \(k[\epsilon]\), and the isomorphism
            \(S[\epsilon]\otimes_{k[\epsilon]} k \cong S\) restricts to an isomorphism
            \(\t{I}\otimes_{k[\epsilon]} k\cong I\).
        \item \(S[\epsilon]/\t{I}\) is an embedded deformation of
            \(S\twoheadrightarrow R=S/I\).
    \end{enumerate}
    \begin{proof}
        (\(1.\Rightarrow 2.\)) Tensoring with $k[\epsilon]/(\epsilon)$ takes
        $S[\epsilon]\to S$ and $\t{I}\to I$, so we get a map
        $S[\epsilon]/\t{I}\to S/I\cong R$ such that
        $(S[\epsilon]/\t{I})\otimes_{k[\epsilon]}k\cong R$. To prove flatness of
        $S[\epsilon]/\t{I}$ over $k[\epsilon]$, we consider the following
        nine-term commutative diagram: 
        \[\begin{tikzcd} 
        & 0 & 0 & 0 \\ 
            0 & {I} & {\t{I}} & I & 0 \\ 
            0 & {S} & {S[\epsilon]} & S & 0 \\ 
            0 & {S/I} & {S[\epsilon]/\t{I}} & R & 0 \\ 
              & 0 & 0 & 0 
              \arrow[from=2-1, to=2-2] 
              \arrow["\epsilon", from=2-2, to=2-3]
              \arrow["\epsilon\mapsto0", from=2-3, to=2-4]
	          \arrow[from=2-3, to=2-4]
	          \arrow[from=2-4, to=2-5]
	          \arrow[from=3-1, to=3-2]
	          \arrow["\epsilon", from=3-2, to=3-3]
	          \arrow["\epsilon\mapsto0", from=3-3, to=3-4]
	          \arrow[from=3-4, to=3-5]
	          \arrow[from=4-1, to=4-2]
	          \arrow["\epsilon", from=4-2, to=4-3]
	          \arrow["\epsilon\mapsto0", from=4-3, to=4-4]
	          \arrow[from=4-4, to=4-5]
	          \arrow[from=1-2, to=2-2]
	          \arrow[from=2-2, to=3-2]
	          \arrow[from=3-2, to=4-2]
	          \arrow[from=4-2, to=5-2]
	          \arrow[from=1-3, to=2-3]
	          \arrow[from=2-3, to=3-3]
	          \arrow[from=3-3, to=4-3]
	          \arrow[from=4-3, to=5-3]
	          \arrow[from=1-4, to=2-4]
	          \arrow[from=2-4, to=3-4]
	          \arrow[from=3-4, to=4-4]
	          \arrow[from=4-4, to=5-4] 
        \end{tikzcd}\] 
        Clearly the columns are exact, and the middle row is exact by flatness
        of \(S\) over \(k\). The top row is right-exact a priori, as we get it
        by tensoring the exact sequence 
        \[0\rightarrow k \xrightarrow{\;\epsilon\;} k[\epsilon] \longrightarrow
        k \rightarrow 0\]
        with $\t{I}$. But it is also left exact, because if $\epsilon x=0$ for
        some $x\in I$, then since $\epsilon x\in S[\epsilon]$, we must have
        $x=0$. Thus by the nine-lemma, the bottom row is exact, so that
        $S[\epsilon]/\t{I}$ is flat over $k[\epsilon]$ by \Cref{flatness}.

        The proof of \((2.\Rightarrow 1.)\) is analogous, using the exactness of
        the bottom row to deduce the exactness of the top row.
    \end{proof}
\end{lemma}

The following result gives a useful classification of embedded deformations.

\begin{theorem} \label{embdef} 
    The embedded deformations of $R$ in $S$ are in natural one-to-one correspondence
    with the set $\Hom_S(I,R)$ of \(S\)-module homomorphisms, with the trivial
    embedded deformation corresponding to the zero map.  
    \begin{proof} 
        Suppose we have an $S$-module homomorphism $\phi:I\to R$. Then define an
        ideal $I_\phi\subset S[\epsilon]$ by 
        \[I_\phi=\bigg\{x+\epsilon y\,\bigg|\,x\in I,y\in S,\phi(x)=y\text{ mod
        }I\bigg\}\] 
        $I_\phi$ is clearly an ideal, as for all $x+\epsilon y\in
        I_\phi$ and $a+\epsilon b\in S$, we have that $( x+\epsilon
        y)(a+\epsilon b)=xa+\epsilon(xb+ay)$. Then since $\phi(x)=y\in R$, we
        have $\phi(xa)=a\phi(x)=ay=ay+xb$ in $R$, so $( x+\epsilon y)(a+\epsilon
        b)\in I_\phi$.  

        Clearly the map $S[\epsilon]\to S$ given by dividing out by $\epsilon S$
        sends $I_\phi$ to $I$. It remains to show that $S[\epsilon]/I_\phi$ is
        flat over $k[\epsilon]$. For this, just replace $\t{I}$ with $I_\phi$ in
        the nine-term commutative diagram on the previous page, and then
        flatness follows by identical reasoning. Note that the zero map
        $\phi=0\in\hom{S}{I}{R}$ gives the ideal $I\oplus\epsilon I$, which is
        the trivial embedded deformation.  

        Conversely, suppose we have an embedded deformation $\t{I}\subset
        S[\epsilon]$. To define our $I\to R$, take any $x\in I$ and lift to an
        element of $\t{I}$. Since $S[\epsilon]\cong S\oplus S\epsilon$, and we
        can write this lift as $x+\epsilon y$ for some $y\in S$. This lift is
        not unique, but differs by an element of $I$: if $x+\epsilon
        y,x+\epsilon y'$ are two lifts, then $\epsilon (y-y')\in \epsilon I$.
        Replacing $I_\phi$ with $\t{I}$ in the previous nine-term commutative
        diagram, we see that since $S[\epsilon]/\t{I}$ is flat over
        $k[\epsilon]$, 
        \[0\longrightarrow I\longrightarrow \t{I}\longrightarrow I\longrightarrow0\] 
        is exact, so that $y-y'\in I$. Thus, the lift $y$ is unique upto an
        element of $I$, giving a well-defined $S$-homomorphism $I\to R$. Note
        that the trivial deformation $I[\epsilon]=I\oplus I\epsilon$ gives the 0
        element in $\hom{S}{I}{R}$, as for any $x\in I$ and lift $x+\epsilon y$,
        we have that $y\in I$, so is zero in $R$.  

        It remains to show that these two correspondences are inverse to one another. So
        given an $S$-homomorphism $\phi:I\to R$, we get an embedded deformation
        $I_\phi$. Then the element of $\Hom_{S}(I,R)$ induced by $I_\phi$ is
        given as follows: fix $x\in I$, then choose any lift $x+\epsilon y\in
        I_\phi$ for some $y\in S$. The induced map is then $x\mapsto y\in R$,
        which is equal to $\phi(x)$. Thus, the homomorphism induced by $I_\phi$
        is the same as $\phi$.  

        Conversely if we have an embedded deformation $S[\epsilon]/\t{I}$, then
        let $\phi:I\to R$ be the corresponding homomorphism. We claim that
        $\t{I}=I_\phi$. Indeed if $x+\epsilon y\in\t{I}$, then $x\in I$ and by
        the definition of $\phi$ we have that $\phi(x)=y+I$, so that $x+\epsilon
        y\in I_\phi$. On the other hand if $x+\epsilon y\in I_\phi$, then
        $\phi(x)=y +I$, so that $x+\epsilon y\in \t{I}$.  
    \end{proof}
\end{theorem}

\begin{remark}
    In fact, any map \(I\rightarrow R\) must factor through the quotient
    \(I\twoheadrightarrow I/I^2\), giving us natural isomorphisms
    \begin{equation} 
        \tag{\(\dagger\)} \label{dagger} \Hom_S(I, R) \cong \Hom_S(I/I^2,
        R)\cong \Hom_R(I/I^2, R).  
    \end{equation} 
    This admits the following geometric interpretation-- whenever
    \(Y\hookrightarrow X\) is a closed subscheme defined by a sheaf of ideals
    \(\mathscr{I}\subset \mathscr{O}_X\), we can define the \textit{normal
    sheaf} of \(Y\) in \(X\) as \[\mathscr{N}_{Y/X} :=
    \sheafHom(\mathscr{I}/\mathscr{I}^2, \mathscr{O}_Y).\] In reasonable cases
    (eg.\ when \(Y\) is a local complete intersection in \(X\)), this is a
    vector bundle (the normal bundle). Then \Cref{embdef} along with
    \eqref{dagger} says that (in the affine case) embedded deformations of
    \(Y\hookrightarrow X\) are given by global sections of
    \(\mathscr{N}_{Y/X}\).  Thus to deform \(Y\) in \(X\), we must `perturb it
    in normal directions'.  
\end{remark}

Deformations in codimension \(1\) have a particularly nice characterisation--
observe that if \(f\in S\) is not a zero-divisor then the ideal \((f)\) is
isomorphic to \(S\) as an \(S\)-module, and hence deformations of
\(S\twoheadrightarrow S/(f)=R\) are in bijection with elements of \(R\).
Moreover, the deformations of principal ideals can be explicitly given by
choosing suitable lifts of elements in \(R\) as follows.

\begin{corollary} \label{principal} 
    Suppose $I\subset S$ is a principal ideal with generator
    \(f\). Then the embedded deformation of \(S\twoheadrightarrow S/I=R\)
    corresponding to \(\phi\in \Hom_S(I,R)\) is given by principal ideal
    \((f+\epsilon f')\subset S[\epsilon]\), where $f'\in S$ is an element such
    that $\phi(f)=f'\mod I$.     
    \begin{proof} 
        In the notation of \Cref{embdef}, if $\phi(f)=f'$ in $R$ then
        $(f+\epsilon f')\subseteq I_\phi$.  To show equality holds, suppose
        $x+\epsilon y\in I_\phi$. Then $x=\alpha f$ for some $\alpha \in S$, and
        $\alpha\phi(f)=y\in I$. Thus, $y=\alpha\phi(f)+\beta f$ for some $f\in
        S$, so that 
        \begin{align*} 
            x+\epsilon y&=\alpha f+\epsilon(\alpha\phi(f)+\beta f) \\ 
                        &=(\alpha+\epsilon\beta) f +\epsilon\alpha\phi(f) \\ 
                        &=(\alpha+\epsilon\beta)(f+\epsilon\phi(f)).  
        \end{align*} 
        Then since $f+\epsilon\phi(f)\in (f+\epsilon f')$, we have that
        $x+\epsilon y\in (f+\epsilon f')$.  
    \end{proof}
\end{corollary}

Using \Cref{embdef} and \Cref{principal}, we can now compute many examples of
embedded deformations. 

\begin{example}[Closed subschemes of \(\mathbb{A}^1_k\)] 
    \label{n points} 
    The ring $S=k[x]$ is a principal ideal domain, so any proper ideal
    $I$ is generated by some polynomial \(f\) of degree \(n\geq 0\). Writing $R=S/I$,
    we see that the closed subscheme $\Spec R\hookrightarrow \Spec S$ consists
    of $n$ points on $\mathbb{A}^1_k$, counted with multiplicity. Then by \Cref{embdef},
    the space of embedded deformations is in bijection with $k[x]/(f)$. But this
    is just the set of all polynomials in $k[x]$ of degree less than $n$, so by
    \Cref{principal} all the embedded deformations of $R$ are of the form 
    \[k[\epsilon,x]/(f+\epsilon (a_{n-1}x^{n-1}+\ldots+a_1x+a_0))\] 
    for $a_0,\ldots,a_{n-1}\in k$. This naturally has the structure of an
    \(n\)-dimensional vector space, thus showing that the Hilbert scheme of \(n\)
    points in \(\mathbb{A}^1_k\) is \(n\)-dimensional.
\end{example}

\subsection{Abstract deformations}

Again, \Cref{def-defthy-abstractscheme} can be rephrased for first-order
deformations of algebras as follows. 

\begin{definition} \label{def-defalgebra}
    An \textit{abstract deformation} of a \(k\)-algebra $R$ is a flat
    $k[\epsilon]$-algebra $\wt{R}$ together with a surjective $k$-algebra
    homomorphism $\beta:\wt{R}\to R$ such that the induced map 
    \[\wt{R}\otimes_{k[\epsilon]}k\longrightarrow R\]
    is an isomorphism.
\end{definition}

Two such deformations $\beta:\wt{R}\to R$ and $\beta':\wt{R}'\to R$ are equivalent if
there is $k[\epsilon]$-algebra homomorphism $\alpha:\wt{R}\to\wt{R}'$ such that
the functor $-\otimes_{k[\epsilon]}k$ takes $\alpha$ to $\id{R}$. More
precisely, there is a commutative diagram of $k$-algebras:
\[\begin{tikzcd}
    \wt{R}\otimes_{k[\epsilon]}k \arrow[r,"\alpha\otimes \text{id}"] \arrow[d,"\cong"] &
    \wt{R}'\otimes_{k[\epsilon]}k \arrow[d,"\cong"] \\
    R \arrow[r, Rightarrow, no head] & R
\end{tikzcd}\]

It is immediate that \(\alpha:\wt{R}\rightarrow \wt{R}'\) is an isomorphism by
applying the five lemma to the following diagram with exact rows.  
\[\begin{tikzcd} 
    0 & R\arrow[d, Rightarrow, no head] & {\widetilde{R}} & R\arrow[d,
    Rightarrow, no head] & 0 \\ 
    0 & R & {\widetilde{R}'} & R & 0 
    \arrow[from=1-1, to=1-2] 
    \arrow["\epsilon", from=1-2, to=1-3]
    \arrow["\beta", from=1-3, to=1-4]
    \arrow[from=1-4, to=1-5]
    \arrow[from=2-1, to=2-2]
    \arrow["\epsilon", from=2-2, to=2-3]
    \arrow["\beta'", from=2-3, to=2-4]
    \arrow[from=2-4, to=2-5]
    \arrow["\alpha", from=1-3, to=2-3] 
\end{tikzcd}\] 

 The \textit{trivial deformation} of a $k$-algebra $R$ is
 $R\otimes_kk[\epsilon]\cong R[\epsilon]$. (To see this is indeed a deformation,
 observe that flatness is preserved by base change and $R$ is automatically flat
 over $k$, so $R[\epsilon]$ is flat over $k[\epsilon]$.
 The equality \((R\otimes_kk[\epsilon])\otimes_{k[\epsilon]}k\cong R\) follows
 from standard commutative algebra.) We will say that a deformation is
 \textit{trivial} if it is equivalent to the trivial deformation, and
 \textit{non-trivial} otherwise. 

\subsubsection{Abstract deformations can be realised in an ambient scheme} 
If we have a surjection \(S\twoheadrightarrow R\), each associated embedded
deformation \(S[\epsilon]\twoheadrightarrow \wt{R}\) gives rise to an abstract
deformation \(\wt{R}\) of \(R\). In fact for a sufficiently large ambient
scheme, all abstract deformations can be realised as embedded deformations in this way.

\begin{proposition}\label{all-defs-are-embedded}
    There is a \(k\)-algebra surjection \(S\twoheadrightarrow R\) such that
    whenever \(\beta:\wt{R}\rightarrow R\) is an abstract deformation of \(R\),
    there is a commutative diagram 
    \[\begin{tikzcd}
        S[\epsilon] \arrow[r, "\epsilon \mapsto 0"] \arrow[d] & S \arrow[d]  \\
        \wt{R} \arrow[r, "\beta"] & R
    \end{tikzcd}\]
    where all the maps are surjections. In particular,
    \(S[\epsilon]\twoheadrightarrow \wt{R}\) is an embedded deformation of
    \(S\twoheadrightarrow R\).
    \begin{proof}
        Let \(S\) be any polynomial algebra \(k[x_i]\) (in possibly infinitely
        many variables) that surjects onto \(R\).
        Abusing notation, we will write \(x_i\) for the image of \(x_i\) in
        \(R\).  

        For each $i$, we can choose a lift $y_i\in\wt{R}$ of $x_i\in R$. Since
        \(S[\epsilon]\) is a free \(k[\epsilon]\)-algebra, we get a map
        \(S[\epsilon]\rightarrow \wt{R}\) sending \(x_i\mapsto y_i\). It remains
        to show this is a surjection-- this follows from the weak five lemma applied
        to the commutative diagram below with exact rows.         
        \[\begin{tikzcd}
            0 \arrow[r]& S \arrow[r,"\epsilon"] \arrow[d, two heads] & S[\epsilon] \arrow[r]
            \arrow[d] & S \arrow[r] \arrow[d, two heads] & 0 \\
            0 \arrow[r]& R \arrow[r, "\epsilon"] & \wt{R} \arrow[r] & R \arrow[r] & 0.
        \end{tikzcd}\] 
        (Surjectivity of the peripheral arrows implies surjectivity of the middle arrow).
    \end{proof}
\end{proposition}

\begin{remark}
There is no canonical embedded deformation inducing a given abstract
deformation, as we had to make a choice of lift of $x_i\in R$ to $y_i\in\wt{R}$.
A different choice of lift, say $y_i'$, would also give an embedded deformation
$S[\epsilon]\to\wt{R}$ sending $x_i\mapsto y_i'$. But using the exact sequence 
\[\begin{tikzcd} 
    0 & R & \wt{R} & R & 0
    \arrow[from=1-1, to=1-2]
    \arrow["\cdot\epsilon", from=1-2, to=1-3]
    \arrow[from=1-3, to=1-4]
    \arrow["", from=1-4, to=1-5]
\end{tikzcd}\]
we deduce that for every $i$ there exists a unique $z_i\in R$ such that
$y_i'=y_i+\epsilon z_i$. Thus, for any two surjections
$\phi,\phi':S[\epsilon]\to\wt{R}$, there is a unique $k[\epsilon]$-algebra
automorphism \(\alpha:\wt{R}\rightarrow \wt{R}\) sending \(y_i\mapsto y_i +
\epsilon z_i\) such that such that $\phi'=\alpha\circ\phi$.
\end{remark}

\subsubsection{Classification of abstract deformations} 
In light of \Cref{all-defs-are-embedded}, to classify abstract deformations of
\(R\) it suffices to find a polynomial algebra \(S\) that surjects onto \(R\)
and then determine which embedded deformations of \(S\twoheadrightarrow R\) are
equivalent as abstract deformations. We shall prove that two embedded
deformations given by \(\phi,\phi'\in \Hom_S(I,R)\) are equivalent if and only
if the difference \(\phi-\phi'\) is induced by a \(k\)-derivation \(S\rightarrow
R\) in the sense that we define below.

\begin{definition} 
    For a \(k\)-algebra \(S\) and an \(S\)-module $M$, a 
    \textit{$k$-derivation} from $S$ into $M$ is a \(k\)-linear map $\delta:S\to
    M$ which satifies the Leibniz rule-- for all $s,s'\in S$,
    \[\delta(ss')=s\delta(s')+s'\delta(s).\] 
    Write \(\text{Der}_k(S,M)\) for the space of \(k\)-derivations from \(S\)
    into \(M\), this is naturally an \(S\)-module.
\end{definition}

In our case, the surjection \(S\rightarrow R= S/I\) induces a natural
\(S\)-module structure on \(R\). Then we can check that the restriction of a
\(k\)-derivation \(\delta:S\rightarrow R\) to \(I\subset S\) is an element of
\(\Hom_S(I,R)\), thus inducing an embedded deformation by \Cref{embdef}. In
fact, we show that these are precisely those embedded deformations that are
trivial as abstract deformations.

\begin{theorem} \label{isom classes} 
    If \(R\cong S/I\) as \(k\)-algebras, then for any
    $\phi,\phi'\in\Hom_{S}({I},{R})$ the corresponding embedded deformations are
    equivalent if and only if $\phi'-\phi$ is the restriction of a
    $k$-derivation \(S\rightarrow R\).  

    \begin{proof} 
        Write \(I_\phi, I_{\phi'}\subset S[\epsilon]\) for the ideals of
        embedded deformations corresponding to \(\phi, \phi'\) respectively. 

        Suppose $\delta:S\rightarrow R$ is a $k$-derivation such that
        \(\phi'(x)-\phi(x)=\delta(x)\) for all \(x\in I\). Then define a map
        \begin{align*} 
            S[\epsilon] &\longrightarrow \wt{R}' \\ x+\epsilon y &\longmapsto x
            + \epsilon y + \epsilon \cdot \delta(x) \mod I_{\phi'}
        \end{align*}
        where we note that \(\epsilon\cdot \delta(x)\) is the image of
        \(\delta(x)\in R\) in the injection \(R\xrightarrow{\epsilon} \wt{R}'\).
        It is clear that this is a homomorphism of \(k\)-algebras, and
        \(I_\phi\) lies in the kernel since the elements of \(I_\phi\) are of
        the form \(x+y\epsilon \in I_\phi\) where \(x\in I\) and \(\phi(x)=y
        \mod I\). Thus we have an induced homomorphism  \(\alpha: \wt{R}\rightarrow
        \wt{R}'\). Moreover, it is clear from the construction that the functor
        \(-\otimes_{k[\epsilon]}k\) (which essentially sets \(\epsilon = 0\))
        sends \(\alpha\) to \(\text{Id}_k\), thus \(\wt{R}\) and \(\wt{R}'\) are
        equivalent as abstract deformations.

        Conversely, suppose we have an equivalence of deformations
        $\alpha:\wt{R}\to \wt{R}'$, then consider the maps
        \[\begin{tikzcd}[row sep = tiny] 
            S[\epsilon]\arrow[r,"\pi", two heads] & \wt{R} \arrow[r,"\alpha"] &
            \wt{R}', S[\epsilon]\arrow[rr,"\pi'", two heads] & & \wt{R}'.
        \end{tikzcd}\] 
        These agree upon applying the functor \(-\otimes_{k[\epsilon]} k\), so
        the image of the difference \((\pi\circ \alpha-\pi')\) must lie in
        \(\epsilon R \subset \wt{R}'\). Thus we can define the composite map
        \[\delta :\quad S \xrightarrow{\;1\mapsto 1\;} S[\epsilon]
            \xrightarrow{\;\alpha \circ \pi - \pi'\;} \epsilon R
            \xrightarrow{\;\epsilon \mapsto 1\;} R.\]
        It can be checked that this is a \(k\)-derivation. For \(x\in
        I\) we have \(\pi(x) +\epsilon \phi(x) =0\) by construction of
        the ideal \(I_\phi\), and likewise \(\pi'(x) + \epsilon \phi'(x) = 0\).
        Moreover, \(\alpha(\epsilon \phi(x)) = \epsilon \phi(x)\), so \(\delta =
        \phi'-\phi\) as required.
    \end{proof}
\end{theorem}

Given a \(k\)-algebra \(R\), choose a surjection \(S\twoheadrightarrow R\) with
kernel \(I\) such that \(S\) is a polynomial algebra. Note any \(R\)-module
\(M\) has a naturally induced \(S\)-module structure, and the ideal \(I\) lies
in the annihilator of \(M\). Then a \(k\)-derivation
\(\delta:S\rightarrow M\) restricts to a homomorphism \(I\rightarrow M\) since 
\[\delta(s\cdot i) = s\cdot \delta(i) + i\cdot \delta(s) = s\cdot \delta(i)\]
whenever \(s\in S\), \(i\in I\). We define the \(R\)-module \(T^1(R/k,M)\) as
the cokernel of the map \(\text{Der}_k(S,M)\to\Hom_{S}(I,M)\) given by
restriction, and write \(T^1_{R/k}\) for \(T^1(R/k,R)\).

Then the result below follows from \Cref{all-defs-are-embedded} and \Cref{isom
classes}.

\begin{corollary} 
    There is a natural one-to-one correspondence between equivalence
    classes of abstract deformations of $R$ and $T^1_{R/k}$.  
\end{corollary}

\begin{remark}
    A priori there is no reason to believe that \(T^1(R/k,M)\) is independent of
    the chosen embedding \(S\twoheadrightarrow R\). But since it turns out to be
    naturally in bijection with the set of abstract deformations of \(R\), it is
    indeed independent of \(S\). In fact, \(T^1\) is a functor which can be
    defined in a more general setting without making choices \cite[see for
    instance][]{hartshorne_deformation_2010}. 
\end{remark}

\subsection{Geometric examples}
If \(S\) is a polynomial algebra surjecting onto \(R\), then we showed how the
elements of \(\Hom_S(I,R)\) naturally correspond to global sections of the
normal sheaf of the inclusion \(\Spec R \hookrightarrow \Spec S \), and thus
deformations of \(\Spec R\) come from perturbing it in normal directions in a
sufficiently large ambient scheme. Which of these are trivial?  

Recall that the module of K\"ahler differentials \(\Omega_{S/k}\) gives the
global sections of the cotangent sheaf on \(\Spec(S)\), and is generated
over \(S\) by all formal symbols \(\{ds\;|\; s\in S\}\). The map \(d:
S\rightarrow \Omega_{S/k}\) given by \(s\mapsto ds\) is a derivation, and is
universal in the sense that for any \(S\)-module \(M\), all \(k\)-derivations
\(S\rightarrow M\) must factor through \(d\) giving a natural isomorphism
\(\Hom_S(\Omega_{S/k}, M) \cong \text{Der}_k(S,M)\). In particular, elements of 
\(\text{Der}_k(S,S)\) are precisely the global sections of the tangent sheaf
(i.e.\ vector fields) on \(\Spec(S)\).

When \(S\) is a polynomial algebra and \(R\cong S/I\), the module
\(\Omega_{S/k}\) is free over \(S\) \cite[Corollary 16.1]{eisen} and hence we
have the exact sequence 
\[0\rightarrow \text{Der}_k(S,I) \rightarrow \text{Der}_k(S,S) \rightarrow
\text{Der}(S,R) \rightarrow 0.\]
Geometrically, this says that \(k\)-derivations \(\delta:S\to R\) are given by 
vector fields on \(\Spec S\) up to fields which vanish on \(\Spec R\) (which
naturally give sections of the normal sheaf). The induced embedded deformation
perturbs \(\Spec R\) along this vector field while preserving the inherent
geometry, thus giving a trivial abstract deformation.

\begin{example}[Zero dimensional schemes]
    Consider $S=k[x]$ and $R=k[x]/(x)$, i.e.\ the inclusion of the origin in
    $\mathbb{A}^1_k$. We have that \(\text{Der}_k(S,R)\) is a one dimensional
    vector space with basis \({\partial}{\partial x}\), and the embedded
    deformation corresponding to the vector field \(a\frac{\partial}{\partial
    x}\) is \(k[\epsilon,x]/(x+a\epsilon)\) which `translates the point with
    speed \(a\)'. In this case the normal bundle (i.e.\ the tangent space to
    \(\mathbb{A}^1_k\) at the origin) has rank \(1\) and thus every embedded
    deformation of \(\Spec R \hookrightarrow \Spec S\) is of the form above,
    showing that \(\Spec R\) has no non-trivial abstract deformations. We say
    that \(\Spec R\) is \textit{infinitesimally rigid}. 

    Consider instead the ordinary double point coming from \(R= k[x]/(x^2)\),
    with the natural surjection from \(S=k[x]\). A \(k\)-derivation
    \(\delta:S\rightarrow R\) given by \(\delta x = ax+b\in R\) must have
    \(\delta(x^2) = 2bx\) and hence the induced deformation is given by the
    ideal \((x^2-2bx\epsilon) = (x-b\epsilon)^2\). This is the trivial
    deformation corresponding to translation along the vector field
    \(b\frac{\partial}{\partial x}\). Note however that \(\Hom_S(I,R)\) is a two
    dimensional vector space, hence an embedded deformation is given by an ideal
    \((x^2-2bx\epsilon + c\epsilon)\subset k[\epsilon, x]\) for \(b,c\in k\).
    Thus we have a one-dimensional family of nontrivial abstract deformations
    \[T^1_{R/k} = \left\{\frac{k[\epsilon,x]}{(x^2-c\epsilon)}\;\mid\; c\in
    k\right\} \] which can be seen as separating the double point into two
    points moving away.  

    The discussion above readily applies to the setting of \Cref{n points}--
    suppose we have a set of \(p\) zero-dimensional subschemes, the \(i\)th one
    having multiplicity \(m_i\). This can be realised as the vanishing locus of
    a degree \(n=\sum_i m_i\) polynomial \(f=\prod_{i}(x-\lambda_i)^{m_i}\in
    k[x]\) for \(\lambda_i\in k\) distinct. We saw that this embedding admits an
    \(n\)-dimensional deformation space parametrised by polynomials of degree at
    most \(n-1\). We should expect the trivial deformations to form a \(p\)
    dimensional subspace coming from translating the `clusters' without changing
    multiplicity. This is indeed the case-- note that
    \[h:=\text{gcd}\left(f,\frac{\partial f}{\partial
    x}\right)=\prod_{i=1}^p(x-\lambda_i)^{m_i-1}\] is a degree \(n-p\)
    polynomial, and by B\'ezout's lemma there exist $\alpha,\beta\in k[x]$ with
    \(h=\alpha f+\beta\frac{\partial f}{\partial x}.\) 

    Given an embedded deformation \((f+\epsilon g)\subset k[x,\epsilon]\), we
    can write \(g=q\cdot h + r\) for a polynomial \(r\in k[x]\) with
    \(\text{deg}(r)<\text{deg}(h)\). Thus we have \[g-r = qh \equiv q\beta
        \frac{\partial f}{\partial x} \pmod{f},\] i.e.\ the deformations
        \((f+\epsilon g)\) and \((f+\epsilon r)\) differ by a trivial
        deformation induced by the derivation \(q\beta \frac{\partial}{\partial
        x}\). Moreover, in this case the map \((f)\to R\) given by \(f\mapsto
        r\pmod f\) is a derivation if and only if \(r=0\), by degree
        considerations. Thus we have a complete description of the equivalence
        classes of deformations of \(\Spec k[x]/(f)\) as \[T^1_{R/k} =
        \left\{\frac{k[\epsilon,x]}{f+\epsilon r}\;|\; \text{deg}(r)<n-p\right\}
    \cong k^{n-p}.\]
\end{example}

\begin{example}[Coordinate axes]
    Consider the ring \(R=k[x,y]/(xy)\). The natural surjection from \(S =
    k[x,y]\) gives an embedding of \(\Spec R\) as the coordinate axes in
    \(\mathbb{A}^2_k\). Then it is clear that embedded deformations of \(\Spec
    R\) are given by ideals \((xy+\epsilon(a+xp(x)+yq(y)))\subset
    k[x,y,\epsilon]\) for polynomials \(p,q\) in \(x,y\) respectively. The map 
    \begin{align*}
        \frac{k[x,y,\epsilon]}{(xy+a\epsilon)}&\rightarrow
        \frac{k[x,y,\epsilon]}{(xy+(a+xp(x)+yq(y))\epsilon)} \\ 
        x &\mapsto x+\epsilon q(y) \\ 
        y &\mapsto y+\epsilon p(x)
    \end{align*}
    is easily seen to be an isomorphism of abstract deformations, induced by the
    derivation \(\delta:=q(y)\frac{\partial}{\partial
    x}+p(x)\frac{\partial}{\partial y}\). Moreover for distinct $a,a'\in k$, the
    corresponding deformations can't be equivalent because there is no way to
    differentiate $xy$ and get $a-a'$. Hence the space of abstract deformations
    is one dimensional, given by the family of conics 
    \[ T^1_{R/k} = \left\{\frac{k[\epsilon,x,y]}{(xy+\epsilon a)} \;\mid \; a\in
    k \right\}.\] 
\end{example}

\begin{example}[Smoothening of hypersurfaces]
    Generalising the above example, consider \(S=k[x_1,...,x_n]\) with a
    principal ideal \(I=(f)\). Then \(R=S/I\) is the coordinate ring of a
    hypersurface, and the embedded deformations of \(\Spec R\) are given by
    \(\Hom_S(I,R)\cong R\). To characterise trivial deformations, note
    \(\Omega_{S/k} =\bigoplus_{i=1}^n S\cdot dx_i\) is free and hence
    \(k\)-derivations \(S\rightarrow R\) are of the form \(\sum_{i=1}^n g_i
    \frac{\partial}{\partial x_i}\) for \(g_i\in R\). Thus the set of trivial
    deformations is precisely the \textit{Jacobian ideal}
    \[
        \left\{\sum_{i=1}^n \frac{\partial f}{\partial x_i}g_i \;|\; g_i \in R
        \right\} =
        \left(\frac{\partial f}{\partial x_1},..., \frac{\partial f}{\partial
        x_n}\right) =: (\nabla f)\subset R.
    \]
    It follows that the abstract deformations of \(\Spec R\) are given by 
    \[T^1_{R/k}=\frac{R}{(\nabla f)} \cong \frac{S}{(f,\nabla f)} \]
    which can be recognised as the coordinate ring of the singular locus of
    \(\Spec R\), often called the \textit{Tjurina algebra} of the hypersurface. 

    \end{example}

\subsubsection{Rigidity of smooth affine varieties} 
The example above shows smooth hypersurfaces are infinitesimally rigid. We shall
see in fact that this characterises smooth affine varieties.  

Suppose \(\Spec R\) is an irreducible affine scheme of finite type embedded in
\(\mathbb{A}^n_k\) via a surjection from \(S=k[x_1,...,x_n]\) with kernel \(I\).
Since $\mathbb{A}^n_k$ is non-singular, we see from \cite[Theorem 8.17]{harts} that
\(\Spec R\) is non-singular if and only if \(\Omega_{R/k}\) is a locally free
\(R\)-module, and the map \(\alpha\) in the conormal sequence
\[I/I^2\overset{\alpha}{\longrightarrow}
\Omega_{S/k}\otimes_SR\to\Omega_{R/k}\longrightarrow 0\] 
is injective. We will see that the functor \(T^1\) provides a
cohomology which measures the failure of the conormal sequence to be left
exact-- in particular detecting if \(\Spec R\) is non-singular.

\begin{proposition}
    For any \(R\)-module \(M\) there is an exact sequence given by
    \[\Hom_R(\Omega_{S/k}\otimes_S R,M)\rightarrow \Hom_R(I/I^2,M) \rightarrow
    T^1(R/k,M)\rightarrow 0.\]
    \begin{proof}
        By the \(\otimes-\Hom\) adjunction, we have natural isomorphisms
        \[\Hom_R(\Omega_{S/k}\otimes_S R, M) \cong
        \Hom_S(\Omega_{S/k},\Hom_R(R,M))\cong\text{Der}_k(S,M)\]  
        where the second isomorphism follows from the definition of
        \(\Omega_{S/k}\) and the fact that \(\Hom_R(R,-)\) is the identity
        functor.  Likewise, we have 
        \[\Hom_R(I/I^2,M)\cong \Hom_S(I/I^2, M)\cong \Hom_S(I,M)\] 
        since \(I\) lies in the annihilator of both \(M\) and \(I/I^2\). It can
        be checked then that the map induced by \(\alpha\) is then simply given
        by restricting the derivation, so that the result follows.
    \end{proof}
\end{proposition}

\begin{theorem} 
    Let $X=\Spec R$ be an irreducible affine scheme of finite type over $k$.
    Then $X$ is non-singular if and only if $T^1({R/k},M)=0$ for all $R$-modules
    $M$, in which case it has no non-trivial deformations.  

    \begin{proof} 
        Since \(\Omega_{R/k}\) is finitely presented, it is locally free if and
        only if it is projective. From the discussion above, it follows that
        \(X\) is non-singular if and only  if the sequence below splits: 
        \[0\rightarrow I/I^2\xrightarrow{\alpha}
        \Omega_{S/k}\otimes_SR\to\Omega_{R/k}\rightarrow 0.\]

        Now if the sequence splits and \(M\) is an \(R\)-module, then the
        sequence remains split-exact after applying the functor \(\Hom_S(-,M)\)
        and hence \(T^1(R/k,M)=0\). 

        Conversely if \(T^1(R/k,M)=0\) for all \(R\)-modules \(M\), then
        choosing \(M=I/I^2\) we see that 
        \[\Hom_R(\Omega_{S/k}\otimes_S R, I/I^2) \rightarrow
        \Hom_R(I/I^2,I/I^2)\] 
        is surjective, so there exists a \(\beta: \Omega_{S/k}\otimes_SR
        \rightarrow I/I^2\) such that \(\beta\circ\alpha=\text{Id}_{I/I^2}\).
        This gives a splitting of the sequence.  
    \end{proof}
\end{theorem}
