\section{Deformation quantisation and \texorpdfstring{$D$}{D}-modules}
\label{section-dmodules}

The deformations examined in \Cref{section-affine-deformation} preserve
commutativity in order to provide a geometric picture. Instead we could also
deform a commutative \(k\)-algebra \(A\) to obtain something possibly non-commutative,
the so-called process of \textit{quantisation}. Classifying flat families of
associative \(k\)-algebras over augmented Artinian rings naturally leads to the
theory of Hochschild cohomology of the special fiber, as is discussed in
\cite{szendroi_unbearable_1999} and \cite{belmans_hochschild_2018}. 

In this section we focus on the special case of \textit{filtered deformations}
of graded rings where the deformation is naturally endowed with the structure of
a Poisson algebra. As a central example, we introduce the theory of
\(D\)-modules, following the exposition in \cite{hotta2007d} with occasional
references to \cite{ginzburg1998lectures} and \cite{bellamy2016noncommutative}.

This theory has its origins in physics, and we retain some of the conventions by
working over the complex numbers and writing \(\hslash\) for the indeterminate.
Thus our deformations are over the powerseries ring \(\mathbb{C}\llbracket
\hslash \rrbracket\) and its Artinian quotients
\(\mathbb{C}[\hslash]/(\hslash^n)\). 


% Given an algebraic variety $X$, the noncommutative algebra $D$ of differential operators on $X$ is a deformation quantisation of the commutative algebra of functions on the cotangent bundle of $X$, $\mathscr{O}_{T^*X}$. Deformation quantisation induces a Poisson bracket on $\mathscr{O}_{T^*X}$ which coincides with that induced by the symplectic structure of $T^*X$. Given a coherent $D$-module one can associate to it a geometric invariant -- its characteristic variety Ch(M), which can be seen to be an involutive subvariety of $T^*X$ with respect to this symplectic structure. If moreover Ch$(M)$ is Lagrangian, that is dim(Ch$(M))$=dim$(X)$, the $D$-module $M$ corresponds to a system of differential equations on $X$ with a finite dimensional solution space.

\subsection{The algebra of differential operators}

Let \(X\) be a smooth algebraic variety over $\mathbb{C}$ of dimension $n$.
Write $\mathscr{O}_X$ for the structure sheaf on $X$, and $TX$ for the
tangent sheaf of $X$, i.e. the sheaf of $\mathbb{C}$-linear derivations on
$\mathscr{O}_X$. Then we construct the sheaf of \textit{differential operators}
on $X$ as the subalgebra of the endomorphisms
$\text{End}_\mathbb{C}(\mathscr{O}_X)$ generated by $\mathscr{O}_X$ and
$TX$. 

More concretely on any open affine neighbourhood $U\in X$ we can choose regular
functions \(x_1,...,x_n \in \mathscr{O}_X(U)\) which satisfy 
\begin{gather*}
    \mathscr{O}_U = \mathbb{C}[x_i,\dots, x_n], \quad
    TU = \bigoplus_{i=1}^n \mathscr{O}_U\cdot\partial_i,
\end{gather*} 
where the variable $\partial_i$ corresponds to the derivation satisfying
$\partial_i(x_j)=\delta_{ij}$. We call $\{x_i, \partial_i\}$ a \emph{local
coordinate system} on $U$. Locally the algebra of differential operators is then the algebra 
\begin{equation*}
    D_U = \bigoplus_{\alpha\in \mathbb{N}^n} \mathscr{O}_U \partial^\alpha,
\end{equation*} where we write \(\partial^\alpha \coloneqq
\prod_{i=1}^n\partial_i^{\alpha_i}\) for tuples \(\alpha = (\alpha_1,...,\alpha_n)\in
\mathbb{N}^n\).

Note that for $f\in \mathscr{O}_U$, a straightforward application of the Leibniz
rule gives us
\[\partial_i(x_j\cdot f)-x_j\cdot\partial_i(f) = \delta_{ij}\cdot f.\]
It can be checked that these are the only independent relations in \(D_U\),
giving us a presentation 
\begin{equation*}
    D_U= \frac{\mathbb{C}[x_1,\dots,x_n, \partial_1, \dots,
    \partial_n]}{\left([\partial_i, x_j] -\delta_{ij}\right)}
\end{equation*} 
where \([\partial_i,x_j] = \partial_ix_j-x_j\partial_i\) is the commutator. In
particular this algebra is noncommutative.

It is straightforward to check that these local constructions are compatible
across affine patches, together giving a sheaf \(D_X\) of
\(\mathscr{O}_X\)-algebras.

\subsubsection{Order filtration}
For a differential operator (i.e.\ a local section of \(D_X\))
\(\partial^\alpha\) given by the \(n\)-tuple \(\alpha=(\alpha_1,...,\alpha_n)\),
we define its order to be \(|\alpha|=\sum_{i=1}^n\alpha_i\). On any affine patch
\(U\), this gives a natural \emph{order filtration} $F$ of the algebra $D_U$ via
\begin{equation*}
    F:\quad 0=F_{-1}D_U\subset F_{0}D_U\subset F_{1}D_U\subset F_{2}D_U \subset\dots,
\end{equation*} 
where for $\ell\in \mathbb{N}$ we define the subspaces $F_\ell D_U \subset D_U$
as
\begin{equation*}
    F_\ell D_U  = \bigoplus_{|\alpha|\leq \ell} \mathscr{O}_U \partial^\alpha.
\end{equation*} 

It is clear from the definition that \(F_0=\mathscr{O}_U\) is the ring of
regular functions on $U$. Moreover the filtration is ascending, i.e. the
containment \(F_\ell D_U \subsetneq F_{\ell+1}D_U\) is strict. Note also that it
is compatible with multiplication in the sense that
\begin{equation*}
    \forall m,n\geq -1, \quad (F_{m}D_U)\cdot (F_n D_U) = F_{m+n} D_U .
\end{equation*}

Less immediate, but very useful fact about this filtration is given by the following proposition.

\begin{proposition}\label{prop:commutators}
    Let $F$ be the order filtration of $D_U$ and consider differential operators
    $P\in F_m D_U$, $Q\in F_n D_U$. The commutator $[P,Q]$ satisfies $[P,Q]\in
    F_{m+n-1}D_U.$ 
    \begin{proof} 
        Follows by induction with base case given by taking $\partial_i\in F_1
        D_U$ and $x_j\in F_0 D_U$. As we have seen before $[\partial_i,
        x_j]=\delta_{ij}\in F_0 D_0 = F_{1+0-1}D_U$ as required.
    \end{proof}
\end{proposition}

\subsubsection{Associated graded ring and principal symbols}
We can use the order filtration $F$ of the algebra $D_U$ to define its
\emph{associated graded ring} as 
\begin{equation*}
    \text{gr}^FD_U=\bigoplus_{\ell=0}^\infty F_\ell D_U / F_{\ell-1} D_U.
\end{equation*}
Multiplication in $\text{gr}^FD_U$ is defined as follows: given elements $a' \in
F_m D_U / F_{m-1} D_U$ and $b' \in F_n D_U / F_{n-1} D_U$ we choose
representatives $a\in F_m D_U$ and $b\in F_n D_U$ and set $a'b'\in F_{m+n} D_U/
F_{m+n-1} D_U$ to be the equivalence class of $ab \in F_{m+n} D_U$. One can
check that this is well defined modulo $F_{m+n-1} D_U$. For a general filtered
algebra, the associated graded algebra is defined analogously.

A nice consequence of \Cref{prop:commutators} is that the ring
\(\text{gr}^FD_U\) is commutative. This allows for techniques from commutative
algebra to be used to study $D_U$ via its associated graded ring.

Writing \(\xi_i = \partial_i \mod F_0 D_U \) for the image of \(\partial_i\) in
\(\text{gr}^FD_U\), we see that 
\begin{equation*}
    F_\ell D_U / F_{\ell-1} D_U = \bigoplus_{|\alpha|=\ell}\mathscr{O}_U \xi^{\alpha},
\end{equation*} 
and hence
\begin{equation*}
    \text{gr}^F D_U = \mathscr{O}_U [\xi_1,\dots\xi_n]=\mathbb{C}[x_1,\dots,x_n,
    \xi_i,\dots,\xi_n]
\end{equation*}  
is simply the polynomial ring in $2n$ variables.

For an order $\ell$ differential operator $P\in F_\ell D_U$ we define its
\emph{principal symbol} as its image in $\mathbb{C}[x_1,\dots,x_n,
\xi_i,\dots,\xi_n]$. Intuitively the principal symbol of a differential operator
$P$ is a polynomial representing $P$ obtained by taking the highest order term
of $P$ and replacing each partial derivative $\partial_i$ by a variable
$\xi_i$.

\subsubsection{The cotangent bundle}
The construction globalises to give a sheaf of graded \(\mathscr{O}_X\)-algebras
\(\text{gr}^FD_X\), whose zeroth graded piece is simply \(\mathscr{O}_X\). The
degree \(1\) principal symbols (corresponding to locally defined differential
operators of order \(1\)) can be identified with vector fields on \(X\). In
other words, the first graded piece is the tangent sheaf \(TX\). Since the 
algebra on each patch is freely generated by its degree \(1\) part, we have a
canonical identification 
\[\text{gr}^FD_X \cong \text{Sym}\,TX = \mathscr{O}_{T\,\dual X}\]
of \(\text{gr}^FD_X\) with the structure sheaf of the cotangent bundle \(T\,\dual
X\). Thus the principal symbols $\xi_i$ correspond to local coordinates $dx_i$
on the cotangent bundle.

\subsection{Deformation quantisation}
We define the notion of a \textit{filtered deformation} of a graded algebra, and
realise the example above as a special case. To avoid working with sheaves, we
work with an smooth affine complex variety \(X\). The local constructions can be
easily shown to glue, and the results generalise to sheaves of algebras on
smooth complex manifolds.

\begin{definition} \label{def-filtered-deformation}
    A \emph{filtered deformation}  of a graded algebra $B$ is a
    filtered algebra $A$, such that the associated graded ring of $A$ satisfies
    $\text{gr}A\cong B$ (as graded algebras). If $B$ is commutative and $A$ is
    noncommutative, $A$ is said to be a \emph{filtered quantisation} or
    \emph{deformation quantisation} of $B$.
\end{definition}

Thus the noncommutative algebra of differential operators $D_X$ on an algebraic
variety $X$ is then a deformation quantisation of the commutative algebra
$\mathscr{O}(T\,\dual X)\cong$ $\text{gr}^F D_X$ of functions on the cotangent
bundle of $X$.

\subsubsection{The Rees algebra} 
Given a commutative $\mathbb{C}$-algebra $B$, we can view a
deformation quantisation of $B$ as an associative $A_{\hslash}$, which
is flat as a $\mathbb{C}\llbracket \hslash\rrbracket$-module, together with an
isomorphism $A_\hslash/(\hslash) \cong B$. The algebra \(A_\hslash\) is
constructed as the \emph{Rees algebra} of \(A\). 

\begin{definition}
    The Rees algebra of a filtered $\mathbb{C}$-algebra $A$ is the
    $\mathbb{C}\llbracket \hslash\rrbracket$-algebra 
    \begin{equation*} 
        A_\hslash = \bigoplus_{l=0}^\infty (F_l A) \hslash^l \subseteq
        A\llbracket\hslash\rrbracket.  
    \end{equation*} 
\end{definition}
It follows from this definition that $A_{\hslash}/(\hslash) \cong \text{gr}^F
A$. Since \(A\) can be recovered from the Rees algebra by looking at its graded
components, the two viewpoints are equivalent. 

In the context of differential operators, write  \(\overline{D_X}\) for the Rees
algebra of \(A=D_X\). Then from the above discussion, this is a deformation
quantisation of $\text{gr}^F D_X\cong \mathscr{O}(T\,\dual X)$. More concretely,
if $D_X$ is generated by the variables $\{x_i, \partial_i\}$, with $x_i \in F_0
D_X$, $\partial_i \in F_1 D_X$, then the Rees ring $\overline{D_X}$ is generated
by $\{x_i, \hslash\partial_i\}$. The commutation relations $[x_i,
\partial_j]=\delta_{ij} \in F_1 D_X$ become $[x_i,
\hslash\partial_j]=\hslash\delta_{ij} \in F_1 D_X \hslash$. Setting
$\xi_i\coloneqq \hslash \partial_i$, the Rees ring is then the
algebra
\begin{equation*}
    \overline{D_X} = \frac{\mathbb{C}[x_1,\dots,x_n, \xi_1\dots,
    \xi_n]}{([\xi_i, x_j] -\hslash\delta_{ij})}.
\end{equation*}
In the limit \(\hslash \rightarrow 0\), this becomes the commutative algebra
\(\mathbb{C}[x_1,...,x_n,\xi_1,...,\xi_n]\).

\subsubsection{The induced Poisson structure}
Deforming a commutative algebra often endows it with additional
structure, namely a \emph{Poisson bracket}. Since the cotangent bundle
of a smooth complex manifold has a natural deformation quantisation given by the
algebra of differential operators, this gives a natural symplectic
structure giving rise to rich geometry.

Recall that a \emph{Poisson algebra} is a commutative, associative unital
algebra $(B, *)$ equipped with a \emph{Poisson bracket} $\{\cdot,\cdot\}:
B\times B\to B$ i.e.\ a Lie bracket which satisfies the Leibnitz identity 
\begin{equation*} 
    \forall a,b,c\in A, \quad \{a,b*c\}=\{a,b\}*c + b*\{a,c\}.
\end{equation*}
Thus for any \(a\in B\), the map \(\{a,\cdot\}:A\to A\) is a derivation. Such a
bracket automatically exists whenever we have a filtered deformation of \(A\).

\begin{proposition}\label{prop:Poisson} 
    If $A$ is a noncommutative filtered algebra such that the associated graded
    ring $B\cong \text{gr}^F A$ is commutative, then $B$ is canonically a
    Poisson algebra.    
    \begin{proof} 
        Given $\tilde{a}\in F_m A / F_{m-1} A$ and $\tilde{b} \in F_n A /
        F_{n-1} A$, we want to construct an element $\{\tilde{a},\tilde{b}\}\in
        \text{gr}^F_A$ such that the axioms of a Poisson bracket are satisfied.
        Consider lifts $a, b \in A$ of the elements $\tilde{a}, \tilde{b}$, and
        recall from \Cref{prop:commutators} that the commutator $ab-ba$ lies in
        $F_{m+n-1} A$. We define the Poisson bracket as
        \[\{\tilde{a},\tilde{b}\} = ab-ba \pmod{F_{m+n-2}A}\in F_{m+n-1} A /
        F_{m+n-2} A.\] We can check that this construction is independent of the
        choice of $a$ and $b$. Since $A$ is associative, and we have defined the
        bracket via the commutator in $A$, it satisfies the axioms of a Poisson
        bracket.  
    \end{proof} 
\end{proposition}  

If \(X\) is a smooth complex algebraic variety, then the Poisson structure on
the cotangent bundle defined locally as above is compatible with gluing and we
get a Poisson bracket on \(\mathscr{O}_{T\,\dual X}\). Thus
the deformation quantisation $D_X$ of $\mathscr{O}(T\,\dual X)$ naturally makes
\(T\,\dual X\) a \textit{Poisson manifold}. It can be checked that this Poisson
bracket induces the canonical symplectic form on \(T\,\dual X\), given 
by \(\omega = \sum d\xi_i\wedge dx_i\) in local coordinates defined above. 

\subsection{An introduction to \texorpdfstring{$D$}{D}-modules}

Let \(X\) be a smooth variety over \(\mathbb{C}\). 

\begin{definition}
    A \emph{$D_X$-module} or simply a \emph{$D$-module} $M$ on \(X\) is a sheaf
    of \(\mathscr{O}_X\)-modules such that on each affine neighbourhood
    $U\subset X$, $M(U)$ is a $D_U$-module in the usual sense and this structure
    is compatible across affine neighbourhoods. Equivalently, a \(D\)-module is
    a sheaf \(M\) of \(\mathscr{O}_X\)-modules with a left action of \(D_X\)
    given by a map of \(\mathscr{O}_X\)-algebras \(D_X\rightarrow
    \text{End}_\mathbb{C}(M)\).
\end{definition}

\begin{example}
    The structure sheaf $\mathscr{O}_X$ is naturally a $D$-module, where the
    action is given by applying the differential operators to functions in
    $\mathscr{O}_X$. This extends to free \(\mathscr{O}_X\)-modules.
\end{example}

\begin{remark}
    \(D_X\) can be seen as the algebra generated by vector fields on \(X\), and
    the commutator coincides with the classical Lie bracket on vector fields.
    Given an element \(v\in D_X\), it is customary to write the induced
    endomorphism of a \(D\)-module \(M\) as \(\nabla_v: M\rightarrow M.\) Then
    these maps satisfy 
    \begin{gather*}
        \nabla_{fv}(m) = f\cdot \nabla_v (m) , \\
        \nabla_v(f\cdot m) = v(f) \cdot m + f\cdot \nabla_v(m),\quad
        \text{and}\\
        \nabla_{[v,w]}(m) = [\nabla_v, \nabla_w] (m)
    \end{gather*}
    for all local sections \(v,w\in D_X\), \(f\in \mathscr{O}_X\), \(m\in M\).
    In particular, if \(M\) is a vector bundle (i.e.\ a locally free sheaf of
    \(\mathscr{O}_X\)-modules) then a \(D\)-module structure on \(M\) is
    precisely a flat connection.
\end{remark}

\subsubsection{Good filtrations} For simplicity we will assume \(X\) is
affine. Let $M$ be a $D_X$-module and let $F$ be the order filtration of $D_X$.
A \emph{compatible filtration} of $M$ is a filtration by finitely generated
\(\mathscr{O}_X\)-modules
\begin{equation*}
    0=F_{-1}M\subset F_0 M \subset F_1 M \subset F_2 M\dots, \quad
    \bigcup_{\ell \geq 0} F_\ell M = M,
\end{equation*}
such that \((F_m D_X)(F_n M)\subseteq F_{m+n}M.\) 
Then the \emph{associated graded module} \(\text{gr}^FM\coloneqq
\bigoplus_{l\geq 0} F_\ell M / F_{\ell-1} M\) is naturally a graded module
over $\text{gr}^F D_X \cong \mathscr{O}_{T\,\dual X}$. 

We call the filtration $F$ a \emph{good filtration} if  $\text{gr}^FM$ is
finitely generated over $\text{gr}^F D_X$. 

The above discussion can be extended to the non-affine case by considering
filtrations on affine patches compatible with the sheaf structure. Recall that
call a sheaf of modules over a sheaf of algebras is \emph{coherent} if it is
locally finitely generated. Then a filtration \(F\) on a \(D\)-module \(M\) is a
\textit{good filtration} if $\text{gr}^FM$ is coherent over the sheaf of
associated graded rings $\text{gr}^F D_X$. 

If \(M\) is coherent as a sheaf of $D_X$-modules, then the submodules \(F_\ell
D_X\cdot M\) are locally finitely generated and we have a canonical good
filtration. In fact, existence of good filtrations is equivalent to coherence.

\begin{theorem}\label{thm:goodfiltration}
    Any coherent $D_X$-module admits a good filtration, and conversely any
    $D_X$-module which admits a good filtration is coherent.
\end{theorem}

\subsubsection{Characteristic variety}

Construction of the sheaf of commutative rings $\text{gr}^FD_X$ from the sheaf
of noncommutative rings $D_X$ allows us to define a geometric invariant of a
$D_X$-module $M$ -- the \emph{characteristic variety}. 

Recall that on an algebraic variety \(X\), the \textit{support} of a sheaf of
\(\mathscr{O}_X\)-modules \(M\) is defined as the subvariety of
points \(\mathfrak{p}\) where the stalk \(M_\mathfrak{p}\) is non-zero. On an
affine neighbourhood \(\Spec R \subset X\), this is the closed subvariety
defined the annihilator of the \(R\)-module corresponding to \(M\).

Now suppose \(X\) is a smooth complex variety with a coherent $D_X$-module \(M\). By
\Cref{thm:goodfiltration}, \(M\) has a good filtration $F$ so that
\(\text{gr}^FM\) is a sheaf of \(\text{gr}^FD_X\cong \mathscr{O}_{T\dual
X}\)-modules. This allows us to naturally think of \(\text{gr}^FM\) as a sheaf
of modules on \(\glSpec \mathscr{O}_{T\,\dual X} \cong T\,\dual X\) where
\(\glSpec\) denotes the global \(\Spec\) construction. 
\begin{definition} 
    The \emph{characteristic variety} (or \emph{singular support}) of
    $M$ is the support of the coherent $\mathscr{O}_{T\,\dual X}$-module
    $\text{gr}^F M$, i.e.\  
    \[\text{Ch}(M) = \text{Supp}\,\text{gr}^F M \quad \subseteq T\,\dual X.\]
\end{definition}

\begin{remark}
    There is a natural action of the multiplicative group
    $\mathbb{C}^\times\coloneqq\mathbb{C}\setminus\{0\}$ on the fibres of
    $T\,\dual X$, and we call a subset $V\subseteq T\,\dual X$ \emph{conic} if
    it is stable under this action. It follows from the construction of the
    characteristic variety that $\text{Ch}(M)$ is a conic subvariety of
    $T\,\dual X$. 
\end{remark}

\begin{example}\label{example-charv-dmodules}
    Let $X=\mathbb{A}^1_x = \Spec \mathbb{C}[x]$, so that $D_X=\mathbb{C}[x,
    \partial]/([x, \partial]-1)$. If $F$ is the order filtration on $D_X$, we
    have $\text{gr}^F D_X=\mathbb{C}[x, \xi]$ where $\xi = \partial \mod F_0
    D_X$. Note that the total space of the cotangent bundle can then be
    identified with the affine space \(\mathbb{A}^2_{(x,\xi)}\), with vector
    space structure coming from projection onto the first coordinate. Then all
    conic subvarieties are of the form \(S\times \{0\}\cup T\times
    \mathbb{A}^1_\xi \subset \mathbb{A}^2_{(x,\xi)}\) for \(S,T\subset
    \mathbb{A}^1_x\).  
    \begin{enumerate}[label=(\roman*)] 
        \item Let $M=0$ be the trivial $D$-module. Then $\text{gr}^FM$ is 0, so
            that $\text{Ann}_{\text{gr}^FD_X} \text{gr}^F M =
            \text{gr}^FD_X$. Thus the characteristic variety
            $\text{Ch}\,(M)=\mathbb{V}(\text{gr}^F D_X)=\varnothing$ is the empty
            space.  
        \item Consider $M=D_X$ seen as a $D$-module over itself with the action
            given by left multiplication. Then the order filtration $F$ of $D_X$
            is trivially a good filtration on the module $M$, and we consider
            the associated graded ring $\text{gr}^F D_X=\text{gr}^F M$ as a
            graded module over itself. The annihilator $\text{Ann}_{\text{gr}^F
            D_X} \text{gr}^F M$ is the $0$ ideal, so that the characteristic
            variety is given by the whole cotangent bundle \(\text{Ch}\,M\cong
            T\,\dual X\).  
        \item Let $M=\mathscr{O}_X=\mathbb{C}[x]$ be the complex polynomial ring
            in one variable seen as a $D$-module under the natural action of
            $D_X=\mathbb{C}[x, \partial]/([x, \partial]-1)$. Thus $x$ acts by
            left multiplication and $\partial$ by differentiation with respect
            to the variable $x$. The filtration 
            \begin{equation*} 
                F_\ell M = \begin{cases} 
                    0 & \ell= -1 \\ 
                    M & \ell \geq 0 
                \end{cases} 
            \end{equation*} 
            is a good filtration of $M$, and the associated graded module
            $\text{gr}^FM$ is isomorphic to $M$ concentrated in degree 0. The
            action of \(\text{gr}^FD_X = \mathbb{C}[x,\xi]\) is such that $x$
            acts by multiplication, but since $\partial\in F_1 D_X$ we have for
            any $m\in M$, \[\xi m = [\partial m] \in F_1 M/ F_0 M = 0 \] i.e.\
            $\xi$ acts by 0 on gr$^F M$. Thus $\text{Ann}_{\text{gr}^F D_X}
            \text{gr}^F M=(\xi)$, and the characteristic variety
            $\text{Ch}\,(M)=\mathbb{V}(\xi)\cong X$ is the zero-section of
            $T\,\dual X$.  
        \item More generally, consider $M= D_X/ D_X(\partial-\lambda)$ for some
            $\lambda\in \mathbb{C}$. This $D$-module can be thought of as being
            associated to the partial differential equation $(\partial-\lambda)f
            = 0$. We construct the associated graded module using the order
            filtration on $D_X$, i.e.\ $F_\ell M = F_\ell D_X\cdot u $ for
            $\ell\geq -1$. Then $\partial-\lambda \in D_X$ acts by 0 on $M$,
            whereas in the associated graded ring $\text{gr}^F D_X$, the class
            $[\partial - \lambda]$ is equal to $\xi$. Thus the characteristic
            variety \(\text{Ch}\,(M)\) is again the zero-section \(X\subset
            T\,\dual X\). Note that we get the same characteristic variety for
            all $\lambda \in \mathbb{C}$.  
        \item  Let $M$ be the $D$-module $M=D_X/D_X(x-a)$ for some $a\in
            \mathbb{C}$, generated by $u= 1 + D_X(x-a)$. The associated partial
            differential equation is $f(x-a)=0$. Consider as above the good
            filtration constructed using the order filtration of $D_X$. The
            associated graded module $\text{gr}^F M$ is generated by the class
            of $u$ in degree 0, and hence the annihilator of $\text{gr}^FM$ is
            precisely the ideal generated by $x-a \in \mathbb{C}[x,\xi]$. Thus
            the characteristic variety
            $\text{Ch}\,(M)=\mathbb{V}(x-a)=\{a\}\times \mathbb{A}^1_\xi$ is the
            $x=a$ fibre inside $T\,\dual X$. This module corresponds to a `delta
            function distribution' supported at $a$.  
    \end{enumerate} 
\end{example}

The following result shows that the characteristic variety is truly a geometric
invariant of a coherent \(D_X\)-module.
\begin{theorem}\label{thm-indep-filtration}
    The characteristic variety of a coherent \(D_X\)-module constructed as above
    is independent of the choice of good filtration on it.
\end{theorem}

\subsubsection{Bernstein inequality} Recall that the
cotangent bundle of any manifold has a canonical symplectic form. If \(X\) is a
smooth complex variety with local coordinates as defined above, this form is
given by \(\omega = \sum d\xi_i\wedge dx_i\) and coincides with the Poisson
bracket on \(\mathscr{O}_{T\,\dual X}\). The characteristic variety of any
coherent \(D_X\)-module is well-behaved with respect to this symplectic
structure, allowing us to estimate its dimension. 

We recall a few notions from symplectic geometry. Let \(V\) be a vector space
with a symplectic form \(\omega:V\otimes V\rightarrow \mathbb{C}\). For any
subspace \(W\subset V\), define \[W^\perp = \{v\in V\,|\, \omega(v,w)=0 \text{
for all }w\in W\}\] and say \(W\) is \textit{involutive} if \(W^\perp \subseteq
W\), and \textit{Lagrangian} if \(W^\perp = W\). Note that the dimension of an
involutive subspace \(W\subset V\) satisfies \[\frac{1}{2}\text{dim}\,V\leq
\text{dim}\,W\leq \text{dim}\, V,\] and the lower bound is attained if and only
if \(W\) is Lagrangian.

If \(X\) is a manifold with a symplectic form \(\omega\), we say a submanifold
\(Y\subset X\) is \textit{involutive} (\textit{Lagrangian}) if the tangent space
\(T_pY\) is an involutive (Lagrangian) subspace of \(T_pX\) for all points
\(p\in Y\). Corresponding constraints on \(\text{dim}\,Y  = \text{dim}\,T_pY\)
hold.

Then we have the following result for a smooth complex variety \(X\).

\begin{theorem}[Sato, Kawai, Kashiwara]
   For any coherent $D_X$-module \(M\neq 0\), the characteristic variety
   $\text{Ch}\,(M)$ is an involutive subvariety of $T\,\dual X$ with respect to
   the canonical symplectic structure.  
\end{theorem}

\begin{corollary}[Bernstein inequality]\label{cor:dimension} 
    If $M$ is a non-zero coherent $D_X$-module, then we have
    \[\textnormal{dim}\,X\leq\textnormal{dim}\,\textnormal{Ch}\,(M)\leq 2\,
    \textnormal{dim}\,X.\] 
\end{corollary}

If \(M\) is a \(D\)-module for which the lower bound in Bernstein inequality is
attained, we say \(M\) is \textit{holonomic}. This is equivalent to
\(\textnormal{Ch}\,(M)\) being a Lagrangian subvariety of \(T\,\dual X\), and we
will see that holonomic \(D\)-modules are well behaved in a number of ways.

\subsection{\texorpdfstring{$D$}{D}-modules and differential equations}
The association between \(D\)-modules and differential equations alluded to in
\Cref{example-charv-dmodules} is made precise by observing that Hom-spaces
between $D$-modules naturally correspond to solution spaces of systems of
differential equations. 

To see this correspondence, consider an open subset $X\subseteq \mathbb{C}^n$
and let $\mathscr{O}_X$ be a ring of functions on $X$
(algebraic/analytic/holomorphic) within which we wish to solve
a system of differential equations. Let $D_X$ be, as before, be the
noncommutative ring of differential operators with coefficients in
$\mathscr{O}_X$. Then $\mathscr{O}_X$ is naturally a left $D_X$-module, where
any $f\in \mathscr{O}_X\subset D_X$ acts by left multiplication and the
variables $\partial_i\in D_X$ act by partial differentiation. 

A differential equation is then determined by a differential operator \(P\in
D_X\), and the solution set we seek is \(\{f\in \mathscr{O}_X\,|\, P\cdot f =
0\}\). This can be phrased in terms of \(D\)-modules as follows.

\begin{proposition} 
    For the left $D$-module $D_X/D_X\cdot P$, we have the natural isomorphism of
    additive groups 
    \begin{equation*} 
        \Hom_{D_X}(D_X/D_X\cdot P, \mathscr{O}_X)\cong \{f\in\mathscr{O}_X
        \, |\, P\cdot f=0\} 
    \end{equation*} 
    \begin{proof} 
        Notice that \(D_X\)-module homomorphisms \(D_X/D_X\cdot P \rightarrow
        \mathscr{O}_X\) are in one-to-one correspondence with \(D_X\)-module
        homomorphisms \(D_X\rightarrow \mathscr{O}_X\) which map \(P\mapsto 0\).
        Moreover any $D_X$-module homomorphism $\psi: D_X\rightarrow
        \mathscr{O}_X$ is determined uniquely by the image $\psi(1)\in
        \mathscr{O}_X$, and we have \(\psi(P) = P\cdot \psi(1)\). Thus we have
        natural isomorphisms
        \begin{align*} 
            \text{Hom}_{D_X}(D_X/D_X\cdot P, \mathscr{O}_X) 
            &\cong \{\psi: D_X\rightarrow \mathscr{O}_X\,|\, P\cdot \psi(1) =
            0\} \\
            &\cong \{f\in \mathscr{O}_X\,|\, P\cdot f = 0\}
        \end{align*}     
        as required.
    \end{proof}
\end{proposition}

A similar description can be given for systems of multiple differential
equations: note that in the above case of a single differential equation $Pf=0$, the
solution space was isomorphic to the space $\Hom_{D_X}(M, \mathscr{O}_X)$ where
the module $M= D_X/ D_X P$ is presented as the cokernel of the morphism $P:
D_X\to D_X$. More generally, a system of $k$ differential equations in $\ell$
unknown functions \((f_1,...,f_\ell)\) can be written as 
\[\sum_{j=1}^\ell P_{ij}f_j = 0,\quad 1\leq i \leq k\]
for some differential operators $P_{ij}\in D_X$. Then the
$\ell \times k$ matrix with entries $P_{ij}$ gives a map $(P_{ij}): D_X^k \to
D_X^\ell$, and a proof similar to above shows the solution space to the system
of equations is isomorphic to $\Hom_{D_X}(M, \mathscr{O}_X)$ where \(M\) is the
cokernel of \((P_{ij})\).

\subsubsection{Holonomic \texorpdfstring{$D$}{D}-modules} A well known fact from
complex analysis is that the space of holomorphic solutions to an ordinary
differential equation in one variable is finite dimensional. This however is not
always the case for partial differential equations on higher dimensional spaces
$X$. We could attempt to classify systems of differential equations which do
admit a finite dimensional space of holomorphic solutions, in light of the
correspondence between $D$-modules and differential equations this is translated
to a classification problem on \(D\)-modules. As we shall see, it is possible
to place constraints on the \(D\)-module \(M\) which ensure finite
dimensionality of the solution space \(\Hom_{D_X}(M,\mathscr{O}_X)\).

Recall that a non-zero coherent \(D\)-module \(M\) on \(X\subset \mathbb{C}^n\)
is holonomic if its characteristic variety has minimal possible dimension (equal
to \(\text{dim}\,X\) by the Bernstein inequality). We build some intuition to
see why constraining the dimension of the characteristic variety is appropriate
to control the dimension of the Hom space.

Consider a system of $k$ differential equations $P_if=0$, $P_i\in D_X$,
$i\in\{1,\dots,k\}$, in a single unknown function $f\in \mathscr{O}_X$. The
$D$-module corresponding to this system sits in an exact sequence
\begin{equation*}
     D_X^k \overset{P}{\longrightarrow} D_X \longrightarrow M \longrightarrow 0
\end{equation*}
where $P : D_X^k\to D_X$ is the map $(Q_1,\dots,Q_2)\mapsto \sum Q_i P_i$. Thus
$M= D_X/ I$, where $I= D_X P_1 +\dots + D_X P_k$. Recall that the characteristic
variety of $M$ is the vanishing set of the annihilator of the associated graded
module gr$^F M$, where in light of \Cref{thm-indep-filtration} we may take $F$
to be the order filtration. Since any differential operator $Q\in
D_X$ gets mapped to its principal symbol $\sigma(Q)\in \text{gr}^FD_X$ under the
natural map, the annihilator of $\text{gr}^F M$ in $\text{gr}^F D_X$ will
consist of the principal symbols $\sigma(Q)$ of differential operators $Q$
belonging to the ideal $I$. It follows that the characteristic variety is
$\text{Ch}\,(M)=\bigcap_{Q\in I}\mathbb{V}(\sigma(Q))\subset T\,\dual X$. 

Now intuitively, the more constrained a system is, the smaller its solution
space. So the more equations we impose, i.e. the larger the ideal $I$, the
better chance we have at obtaining a finite dimensional solution space. But the
larger is $I$, the more principal symbols we have in the annihilator of
$\text{gr}^F M$, so the smaller is their vanishing set $\text{Ch}\,(M)$.
Therefore it would make sense to require the dimension of Ch$(M)$ to be small
for the space of solutions to the system corresponding to $M$ to be finite
dimensional. This is indeed true.

\begin{theorem}
    Let $M, N$ be holonomic left $D_X$-modules. Then the space $\Hom_{D_X}(M, N)$
    is finite dimensional.
\end{theorem}

We have already seen in \Cref{example-charv-dmodules} that \(\mathscr{O}_X\) is
holonomic (its characteristic variety is the zero-section \(X\)). Thus in
particular, when $M$ is a holonomic $D$-module corresponding to a system of
differential equations $P$, the solution solution space of $P$ is finite
dimensional. It is for this reason that holonomic modules are often called
\emph{maximally overdetermined}.
