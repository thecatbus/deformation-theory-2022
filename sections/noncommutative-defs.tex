\section{Non-commutative deformations of a point}
\label{section-noncommutative-def}
Given a \(k\)-scheme \(X\), the results of \Cref{subsec-defthy-of-a-point}
suggest that the local properties of \(X\) can be inferred from the deformation
theory of its skyscraper sheaves \(\mathscr{O}_{X,x}\). In this section, we work
with the more general situation of a (left) module \(M\) over an
associative \(k\)-algebra \(A\). 

If \(M\) is one dimensional over \(k\), then it is determined by a maximal left
ideal \(p\subset A\) corresponding to the kernel of the map \(A\rightarrow
\text{End}_A(M)\). If \(A\) is commutative (so that \(M\) corresponds to the
skyscraper sheaf of a point) then \Cref{prop-defthy-of-a-point} shows the the
functor \(\Def_M:\text{cArt}^1_k \rightarrow \text{Set}\) is prorepresented by
the completion \(\widehat{A}_p\).  Of course when \(A\) is not commutative, the
best we can hope to recover from the classical deformation problem is an
abelianisation of \(\widehat{A}_p\).  This indicates that we ought to consider
instead the \textit{non-commutative deformation theory} of \(M\), where the test
objects lie in \(\text{Art}^1_k\), the category of all Artinian local
\(k\)-algebras with residue field \(k\).

\begin{definition} 
    Let \(M\) be any \(A\)-module. For an Artinian \(k\)-algebra \(R\in
    \text{Art}^1_k\),  we define an \textit{\(R\)-deformation of \(M\)} to be a
    left \(A\otimes R^{\text{op}}\)-module \(M_R\) that is free over \(R\), with
    an isomorphism of \(A\)-modules \(M_R\otimes_R k \rightarrow M\). Two such
    deformations \(M_R\), \(M_R'\) are equivalent if there is an isomorphism
    \(M_R\rightarrow M_R'\) of \(A\otimes R\)-modules compatible with the
    identifications
    \[M_R\otimes_R k \rightarrow M \leftarrow M_R'\otimes_R k.\] 
\end{definition}

Write \(\Def_M\) for the resulting deformation functor
\(\text{Art}^1_k\rightarrow \text{Set}\), which maps \(R\) to the set of
\(R\)-deformations of \(M\) up to equivalence. Restricted to the full
subcategory \(\text{cArt}^1_k\subset \text{Art}^1_k\), this is the classical
deformation functor described in \Cref{subsec-various-deformation-problems}.

\subsubsection{Deforming via the structure map} We can rephrase the problem in
terms of deformations of the natural \(k\)-algebra map  \(\mu: A\rightarrow
\text{End}_k(M)\) as follows. An \(R\)-deformation \(M_R\) of \(M\) it is free
over \(R\) hence is completely determined by its \(A\)-module structure, given
by a map
\[
    \mu_R: A\rightarrow \text{End}_{k}(M)\otimes R\quad(\cong \text{End}_R(M\otimes R))
\]
such that \(\mu_R\) reduces modulo \(\mathfrak{m}_R\) to the map \(\mu\). Two
such maps \(\mu_R,\mu_R':A\rightarrow \text{End}_k(M)\otimes R\) are equivalent
if they differ by an inner \(k\)-algebra automorphism of
\(\text{End}_k(M)\otimes R\). 

\begin{proposition}\label{lemma-prorep-by-completion}
    If \(M\) is a one-dimensional \(A\)-module, then for \(R\in \textnormal{Art}^1_k\)
    there are natural isomorphisms
    \begin{align*} 
        \Def_M(R) &= \Hom_{\textnormal{Alg}^1_k}(A,R)/\{\text{Inner automorphisms of
        \(R\)}\}\\ 
                  &= \Hom_{\textnormal{Alg}^1_k}(\widehat{A}_p,R)/\{\text{Inner
                  automorphisms of \(R\)}\}
    \end{align*} 
    where \(\widehat{A}_p=\overset{\lim}{\rightarrow} A/p^n\) is the completion of
    \(A\) at \(p=\ker \mu\).     
    \begin{proof} 
        Since \(M\) is one dimensional, we have \(\text{End}_k(M)\otimes R \cong
        R\) and so deformations of \(\mu\) are given by maps
        \(\mu_R:A\rightarrow R\) that reduce modulo \(\mathfrak{m}_R\) to
        \(\mu\), up to inner automorphisms of \(R\). But considering \(\mu\) to
        be an augmentation of \(A\), this is equivalent to saying \(\mu_R\) is a
        map of augmented \(k\)-algebras. Thus we have \[\Def_M(R) =
        \Hom_{\textnormal{Alg}^1_k}(A,R)/\{\text{Inner automorphisms of
    \(R\)}\}.\] For the second isomorphism, note that \(R\) is Artinian so any
    map \(A\rightarrow R\) that respects the augmentation should factor through
    \(A/p^n\) for sufficiently large \(n\).
    \end{proof}
\end{proposition}

\begin{remark}
    If we restrict to \(\text{cArt}^1_k\), then \(R\) has no nontrivial inner
    automorphisms. Moreover, any map \(A\rightarrow R\) must factor uniquely
    through the surjection \(A\rightarrow A_\text{Ab}\) whose kernel is the
    ideal generated by commutativity relations in \(A\). Thus the restriction of
    \(\Def_M\) to \(\text{cArt}^1_k\) is prorepresented by
    \(\widehat{A}_{\text{Ab},p}\), the completion of \(A_\text{Ab}\) at the
    image of \(p=\ker\mu\). If \(A\) is commutative, this gives another proof of
    \Cref{prop-defthy-of-a-point}. 
\end{remark}

\subsection{Deformations by lifting resolutions}
\label{subsubsec-defbylifting}
Following the treatment of \cite{eriksen_noncommutative_2017}, we give an
explicit description of the deformation functor \(\Def_M\) in the general case
by choosing a a projective resolution 
\[\cdots \rightarrow P_2 \xrightarrow{\;d_1\;} P_1 \xrightarrow{\;d_0\;} P_0
\rightarrow M.\]

\begin{definition} 
    For \(R\in \text{Art}^1_k\), an \textit{\(R\)-lift} of the complex
    \((P_\bullet, d)\) is a complex \((P_\bullet\otimes R, \partial)\) of
    \(A\otimes R^{\text{op}}\)-modules that reduces modulo \(\mathfrak{m}_R\) to
    \((P_\bullet, d)\).  Two such liftings are equivalent if there is a chain
    isomorphism \((P_\bullet\otimes R, \partial) \rightarrow (P_\bullet\otimes
    R, \partial')\) inducing the identity map on \((P_\bullet, d)\).
\end{definition}

We now show that isomorphism classes of such \(R\)-lifts are in bijection with
the set \(\Def_M(R)\).

\begin{lemma}
    If \((P_\bullet\otimes R, \partial)\) is an \(R\)-lift of the
    complex \((P_\bullet, d)\) defined above, then \((P_\bullet\otimes
    R, \partial)\) is a projective resolution of the \(A\otimes
    R^{\text{op}}\)-module \(M_R=\coker\partial_0\) which is naturally an
    \(R\)-deformation of \(M\).

    \begin{proof}
        It is clear that each \(P_\bullet\otimes R\) is a projective \(A\otimes
        R\)-module. To show that \((P_\bullet\otimes R, \partial)\) is
        indeed a resolution of \(\coker\partial_0\), we use the fact that the
        surjection \(R\rightarrow k\) in \(\text{Art}^1_k\) can be factored
        as 
        \[R=R_n\xrightarrow{u_n}R_{n-1}\rightarrow ...
        \xrightarrow{u_1}R_0=k\] 
        where each \(u_i\) is a surjection with
        one-dimensional kernel (such maps are called \textit{small
        surjections}). This gives us surjective chain maps
        \[(P_\bullet\otimes R,\partial) = (P_\bullet\otimes
            R_n,\partial^n)\rightarrow (P_\bullet\otimes
            R_{n-1},\partial^{n-1}) \rightarrow \cdots \rightarrow
            (P_\bullet\otimes R_0,\partial^0)=(P_\bullet, d)\]
        where each map is reduction modulo \(\ker u_i\). It is immediate that
        each \((P_\bullet\otimes R_i,\partial^i)\) is an \(R_i\)-lift of
        \((P_\bullet, d)\). Writing \(K:=\ker u_i\), note we have a
        short exact sequence of chain complexes 
        \[\begin{tikzcd} 
            \cdots  & 0\arrow[d] & 0\arrow[d] & 0\arrow[d] & 0\arrow[d] & \\
            \cdots \arrow[r] & P_2\otimes K \arrow[r]\arrow[d] &
            P_1\otimes K\arrow[r]\arrow[d] &
            P_0\otimes K\arrow[r]\arrow[d] & N\arrow[r]\arrow[d] & 0 \\ 
            \cdots \arrow[r] & P_2\otimes R_i \arrow[r] \arrow[d] & P_1\otimes
            R_i \arrow[r]\arrow[d] & P_0\otimes R_i \arrow[r]\arrow[d] & M_{R_i}
            \arrow[r]\arrow[d] & 0 \\
            \cdots \arrow[r] & P_2 \otimes R_{i-1} \arrow[r]\arrow[d] &
            P_1\otimes R_{i-1}\arrow[r]\arrow[d] &
            P_0\otimes R_{i-1}\arrow[r]\arrow[d] & M_{R_{i-1}}\arrow[r]\arrow[d]
                                                 & 0 \\ 
            \cdots  & 0 & 0 & 0 & 0 & 
        \end{tikzcd}\]
        where the maps \(P_n\otimes K\rightarrow P_{n-1}\otimes K\) are the
        restrictions of \(\partial_n^i\). Since \(K\) is one dimensional, it is
        generated by some \(r\in R_i\). Then for any \(p\otimes r\in
        P_n\otimes K\), we have 
        \begin{align*} 
            \partial^i_n (p\otimes r) &= r\cdot \partial_n^i(p\otimes 1)\\
                                    &= r\cdot (d_np\otimes 1 + q) \\ 
                                    &= d_np \otimes r +  r\cdot q
        \end{align*}
        for some \(q\in P_{n-1}\otimes \mathfrak{m}_{R_i}\).
        But \(K\) is a one-dimensional \(R_i\)-module, so must be annihilated
        by \(\mathfrak{m}_{R_i}\). Thus \( r\cdot q = 0\),
        and we have \(\partial^i_n(p\otimes r) = d_n p \otimes r\). In other
        words, the complex \((P_\bullet \otimes K,\partial^i)\) is
        isomorphic to \((P_\bullet, d)\). Taking the long exact sequence
        on homology, we see that for all \(j>0\)
        \[H_j(P_\bullet\otimes R_n, \partial^n) \cong
        H_j(P_\bullet\otimes R_{n-1}, \partial^{n-1})  \cong \cdots
        \cong H_j(P_\bullet\otimes R_{0}, \partial^{0}) \cong 0\]
        i.e.\ \((P_\bullet \otimes R, \partial)\) is a projective
        resolution of \(M_R=\coker \partial_0\). 

        It is clear that \(M_R\) reduces modulo \(\mathfrak{m}_R\) to
        \(M\). An inductive argument using the short exact sequence above shows
        \(M_R\) is free over \(R\), hence is an \(R\)-deformation of
        \(M\).
    \end{proof}
\end{lemma}

\begin{lemma}
    If \(M_R\) is an \(R\)-deformation of \(M\), then the complex
    \((P_\bullet,d)\) has an \(R\)-lift \((P_\bullet\otimes R,
    \partial)\) such that \(\coker \partial_0 \cong M_R\).
    \begin{proof}
        It suffices work with small surjections. Suppose \(R\rightarrow S\) is
        a small surjection with kernel \(K\), and \(M_S\) is the reduction
        modulo \(K\) of \(M_R\) such that the \(S\)-lift \((P_\bullet\otimes S,
        \partial'_\bullet)\) satisfies \(M_S\cong \coker \partial'_0\). It is
        clear that the sequence 
        \[0\rightarrow M\otimes K \rightarrow M_R \rightarrow M_S
        \rightarrow 0\] 
        is exact. Then consider the diagram
        \[\begin{tikzcd} 
            \cdots  & 0\arrow[d] & 0\arrow[d] & 0\arrow[d] & 0\arrow[d] & \\
            \cdots \arrow[r, dashed] & P_2\otimes K \arrow[r,dashed]\arrow[d] &
            P_1\otimes K\arrow[r,dashed]\arrow[d] &
            P_0\otimes K\arrow[r,dashed]\arrow[d] & M\otimes K\arrow[r,
            dashed]\arrow[d]
                                                  & 0 \\ 
            \cdots \arrow[r, dashed] & P_2\otimes R \arrow[r, dashed] \arrow[d]
                                     & P_1\otimes R \arrow[r, dashed]\arrow[d] &
            P_0\otimes R \arrow[r,dashed]\arrow[d] & M_{R}
            \arrow[r,dashed]\arrow[d] & 0 \\
            \cdots \arrow[r] & P_2 \otimes S \arrow[r]\arrow[d] &
            P_1\otimes S\arrow[r]\arrow[d] &
            P_0\otimes S\arrow[r, "\rho_S"]\arrow[d] & M_S\arrow[r]\arrow[d]
                                                 & 0 \\ 
            \cdots  & 0 & 0 & 0 & 0 & 
        \end{tikzcd}\]
        with exact columns. Since \(P_0\) is projective, we can lift \(\rho_S\)
        to a morphism \(\rho_R: P_0\otimes R \rightarrow M_R\). If \(K\) is
        generated by \(r\), we see that for any \(p\otimes r \in P_0\otimes K\)
        we have \(\rho_R(p\otimes r) = r\cdot \rho_R(p\otimes 1)\). But writing
        \(\rho: P_0\rightarrow M\) for the natural map, we have 
        \begin{align*}
            \rho(p) &= \rho_S(p\otimes 1) \mod \mathfrak{m}_S \\ &=
            \rho_R(p\otimes 1)\mod \mathfrak{m}_R.
        \end{align*}
        Combined with the fact that \(\mathfrak{m}_RK=0\), this implies
        \(\rho_R(p\otimes r) = \rho(p)\otimes 1\). In particular the map
        \(P_0\otimes K \rightarrow M\otimes K\) is surjective, so by the Snake
        lemma we have a surjection \(\ker(\rho_R)\rightarrow \ker(\rho_S)\). 
        This allows us to inductively lift the differentials
        \(\partial'_n\), getting a lift \((P_\bullet\otimes R,
        \partial)\) as required.
    \end{proof}
\end{lemma}

It is clear from the above constructions that equivalent deformations of \(M\)
correspond to equivalent lifts of \((P_\bullet, d)\). Thus
the deformation functor can be described as
\[\Def_M(R) = \{\text{\(R\)-lifts of the complex \((P_\bullet, d)\)
up to isomorphism}\}.\]

\subsubsection{A hint of Maurer-Cartan} Consider the graded \(A\)-module
\(P:=\bigoplus_i P_i\). All \(R\)-lifts of \((P_\bullet, d)\) have the
underlying graded \(A\otimes
R^{\text{op}}\)-module \(P\otimes R\) so an \(R\)-lift is completely determined
by its differential \(\partial\), seen as a degree \(1\) endomorphism of
\(P\otimes R\) which satisfies \(\partial \circ \partial = 0\)
and reduces modulo \(\mathfrak{m}_R\) to \(d\). 

Observe we have the isomorphism of graded modules
\[\text{End}_{A\otimes R}(P\otimes R) \cong \text{End}_A(P)\otimes R.\]
Using the splitting \(R=k\oplus \mathfrak{m}_R\) we see
that \(\partial\) is determined by the degree \(1\) element
\(\delta := \partial - d \in \text{End}_A(P)\otimes
\mathfrak{m}_R.\) The condition \(\partial \circ \partial = 0\)
is equivalent to 
\begin{equation} \label{pre-maurercartan}
    (d\circ \delta + \delta \circ d) +
    \delta \circ \delta = 0.
\end{equation}
Moreover, it is clear that any degree \(1\) element \(\delta\in
\text{End}_A(P)\otimes \mathfrak{m}\) satisfying \eqref{pre-maurercartan}
determines an \(R\)-lift of \((P_\bullet,d)\). 

Two such elements \(\delta, \delta'\) determine equivalent lifts
if and only if there is a degree \(0\) element \(\varphi\in
\text{End}_A(P)\otimes \mathfrak{m}_R\) such that the map \(1+\varphi\in
\text{End}_A(P)\otimes R\) is an isomorphism of chain complexes
\((P_\bullet,d+\delta)\rightarrow (P_\bullet, d +
\delta')\). Since \(R\) is Artinian, \(1+\varphi\) is already
invertible as map of graded modules, and we require \(\varphi\) to satisfy
\begin{equation} \label{pre-gaugeaction}
    \delta' = \left((1+\varphi)\circ\delta -
    (d \circ \varphi - \varphi \circ d)\right)
    \circ (1+\varphi)^{-1}.
\end{equation}
Any degree \(0\) element \(\varphi\in \text{End}_A(P)\otimes
\mathfrak{m}_R\) satisfying \eqref{pre-gaugeaction} gives an equivalence of lifts
between \(\delta\) and \(\delta'\). Thus we have a description
of the deformation functor
\[\Def_M(R) = \frac{\{\text{degree \(1\) elements of
\(\text{End}_A(P)\otimes \mathfrak{m}_R\) satisfying
\eqref{pre-maurercartan}}\}}{\{\text{equivalences given by
\eqref{pre-gaugeaction}}\}}.\]

\subsection{Deformation functors from differentially graded algebras}
\label{subsec-defbydga}
We provide a succinct reformulation of  the results of
\Cref{subsubsec-defbylifting} in terms of \textit{differentially graded Lie algebras}. 

\begin{definition} A differentially graded Lie algebra (DGLA) \(\Gamma\) 
    is given by a cochain complex \((\Gamma^\bullet,\underline{d})\) of \(k\)-vector
    spaces with a bilinear graded bracket \([\cdot,\cdot]: \Gamma^m \times \Gamma^n
    \rightarrow \Gamma^{m+n}\) satisfying the following compatibilities for
    homogeneous elements \(g_i\in \Gamma^{m_i}\): 
    \begin{gather*}
        \underline{d}[g_1,g_2] = [\underline{d}g_1,g_2] + (-1)^{m_1}
        [g_1,\underline{d}g_2],\\
        [g_1,g_2] + (-1)^{m_1m_2}[g_2,g_1]=0,\\
        [g_1,[g_2,g_3]+(-1)^{m_3(m_1+m_2)}[g_3,[g_1,g_2]] +
        (-1)^{m_1(m_2+m_3)}[g_2,[g_3,g_1]]=0.
    \end{gather*}
    We say \(\Gamma\) is \textit{nilpotent} if its central series
    \(\Gamma\supset [\Gamma,\Gamma]\supset [\Gamma,[\Gamma,\Gamma]]\supset ...\)
    stabilises to zero.  

    \(\Gamma\) has an associated \textit{Maurer-Cartan equation}, the solutions
    of which form the set of \textit{Maurer-Cartan elements} \[MC(\Gamma) =
    \left\{x\in \Gamma^1\;|\; \underline{d}x+\frac{1}{2}[x,x]=0\right\}.\]
\end{definition}

\subsubsection{Deformations from Maurer-Cartan elements} For an \(A\)-module
\(M\), we construct a DGLA whose Maurer-Cartan elements naturally give
deformations of \(M\). Fix a projective resolution \((P_\bullet,d)\) of \(M\)
with underlying graded module \(P=\bigoplus_i(P_i)\), so that
\(\text{End}_A(P)\) is a graded algebra with multiplication given by
composition. Then the degree one map \(\underline{d}:\text{End}_A(P)\rightarrow
\text{End}_A(P)\) given on homogeneous elements by 
\[\underline{d}\varphi = d \circ \varphi
-(-1)^{\text{deg}\varphi} \varphi \circ d\]
satisfies the graded Leibniz rule, giving \(\text{End}_A(P)\) it the structure
of a differential graded algebra (DGA). To get a differential graded Lie algebra, we
take the commutator bracket 
\[[\varphi, \vartheta] = \varphi\circ \vartheta -
(-1)^{\text{deg}\varphi\cdot \text{deg}\vartheta}\,\vartheta \circ \varphi.\]
The bracket and the differential are carried over to \(\text{End}_A(P)\otimes
\mathfrak{m}_R\), giving us a DGLA whose Maurer-Cartan elements are
precisely the solutions to \eqref{pre-maurercartan}. 

Observe that the DGLA \(\text{End}_A(P)\otimes \mathfrak{m}_R\) is
always nilpotent, giving the degree \(0\) component
a well-defined group operation via the Baker-Campbell-Hausdorff formula 
\[\varphi \ast \vartheta = \varphi + \vartheta
+ \frac{1}{2}[\varphi, \vartheta]+\frac{1}{12}[\varphi,
[\varphi, \vartheta]]+\cdots.\] 
This is the \textit{gauge group} of the DGLA, written
\(G(\text{End}_A(P)\otimes\mathfrak{m}_R)\). It naturally acts on the set
\(MC(\text{End}_A(P)\otimes \mathfrak{m}_R)\) via a \textit{gauge action} which
is the exponential of the Lie algebra action. An explicit check shows that
\(\varphi\in G(\text{End}_A(P)\otimes \mathfrak{m}_R\) identifies two
Maurer-Cartan elements \(\delta,\delta'\) via this action if and only if the
relation \eqref{pre-gaugeaction} holds, so we see that
\[\Def_M(R) = \frac{MC(\text{End}_A(P)\otimes\mathfrak{m}_R)}
{G(\text{End}_A(P)\otimes\mathfrak{m}_R)}.\]

\begin{example}
    If \(R=k[\epsilon]\) is the ring of dual numbers, then we see that the
    Maurer-Cartan equation becomes \(\underline{d}\delta = 0\). Thus Maurer-Cartan elements
    of \(\text{End}_A(P)\otimes \mathfrak{m}_R\) are given by \(\delta = f \otimes
    \epsilon\) where \(f: P_\bullet\rightarrow P_\bullet\) satisfies 
    \(f\circ d = d\circ f\), i.e.\ \(f\) is a chain map.

    Suppose the degree zero map \(h:P_\bullet\rightarrow P_\bullet\) is such
    that the gauge action of \(h \otimes \epsilon \) identifies \(f\otimes
    \epsilon\) with \(g \otimes \epsilon\). Then we have 
    \(g = f + d \circ h- h \circ d\) i.e.\ \(h\) is a chain homotopy between
    \(f\) and \(g\). Thus the tangent space of \(\Def_M\) is given by the space
    of homotopy classes of degree \(1\) chain maps \(P_\bullet \rightarrow
    P_\bullet\), i.e.\
    \[\Def_M(k[\epsilon]) = \Ext^1_A(M,M),\]
    \end{example}

\begin{remark}
    Both \(MC(-)\) and \(G(-)\) can be seen as functors on the category of
    nilpotent DGLAs, and the action of the Gauge group on the solutions of the
    Maurer-Cartan equation respects this functoriality. This gives a way to
    associate a deformation functor to any differentially graded lie algebra
    \(\Gamma_\bullet\), via 
    \[R\mapsto \frac{MC(\Gamma_\bullet\otimes
    \mathfrak{m}_R)}{G(\Gamma_\bullet\otimes \mathfrak{m}_R)}.\]
    \cite{szendroi_unbearable_1999} gives an introduction to Deligne's philosophy that
    every deformation problem (in characteristic zero) must arise from a DGLA in
    this fashion. 
\end{remark}

\subsection{Deformation functors from \texorpdfstring{\(A_\infty\)}{Ainfty}-algebras}
The construction of \Cref{subsec-defbydga}, while functorial, leaves us to deal
with the infinite dimensional chain complex \(\text{End}_A(P)\) given in degree
\(n\) by \(\prod_{i\in \mathbb{Z}}\Hom_A(P_{i+n},P_i)\). It is standard that the
deformation functor only depends on the differentially graded (Lie) algebra up
to quasi-isomorphism, so we can hope to replace \(\text{End}_A(P)\) by a complex
which has finite dimensional components.  The obvious
candidate is the cohomology complex \(\Ext_A(M,M)\) with the induced
multiplication (i.e.\ the Yoneda product), but often there is no
quasi-isomorphism of DGAs between the two-- too much information is lost when
passing to homology. However, the process of \textit{homological perturbation}
allows us to put additional structure on \(\Ext_A(M,M)\) which does allow one to
reconstruct the original DGA.

\subsubsection{A brief introduction to \(A_\infty\)algebras} The structure
alluded to above is that of an \textit{\(A_\infty\) algebra}, which is given by
infinitely many multilinear maps on the complex which can be interpreted as
`higher homotopies' \cite[see][Appendix A for the geometric
motivation]{segal__2008}.

\begin{definition} 
    An \textit{\(A_\infty\) algebra} is a \(\mathbb{Z}\)-graded vector space
    \(V=\bigoplus_i V_i\) with a sequence of linear maps \(m_n : V^{\otimes
    n}\rightarrow V\) for \(n\geq 1\) such that \(m_n\) has degree \(2-n\), and
    satisfies
    \begin{equation}\label{Ainftymult}
        \sum_{n=r+s+t}(-1)^{r+st}m_{r+1+t}(\mathbold{1}^{\otimes r}\otimes m_s
        \otimes \mathbold{1}^{\otimes t}) = 0.
    \end{equation}
\end{definition}

Thus we have \(m_1\circ m_1=0\) i.e.\ \((V,m_1)\) is a cochain complex, and
\(m_1\circ m_2 = m_2(m_1\otimes \mathbold{1} + \mathbold{1}\otimes m_1)\) i.e.\
\(m_2\) is a product satisfying the Leibniz rule. The higher multiplications
give homotopies up to which \(m_2\) is associative. 

A morphism of \(A_\infty\) algebras \(f:V\rightarrow W\) is likewise given by a
sequence of maps \(f_n: V^{\otimes n}\rightarrow W\) for \(n\geq 1\), such that
\(f_n\) has degree \(n-1\) and satisfies 
\begin{equation} \label{Ainftymorphism}
    \sum_{n=r+s+t}(-1)^{r+st}\,f_{r+t+1}(\mathbold{1}^{\otimes r}\otimes m_s
    \otimes \mathbold{1}^{\otimes t}) =
    \sum_{n=i_1+...+i_r}(-1)^{\sum_j(r-j)(i_j-1)}
    m_r(f_{i_1}\otimes...\otimes f_{i_r}). 
\end{equation}
Thus \(f_1\circ m_1 = m_1 \circ f_1\) i.e.\ \(f_1\) is a chain map, and
\(f_1\circ m_2 = m_2(f_1\circ f_1) + m_1f_2 + f_2(m_1\otimes \mathbold{1} +
\mathbold{1}\otimes m_1)\) i.e.\ \(f_1\) comutes with \(m_2\) up to a homotopy
given by \(f_2\). The higher components of \(f\) can be interpreted as
homotopies up to which lower \(f_i\)s are compatible with the multiplications.

We say \(f\) is a quasi-isomorphism if \(f_1\) is a quasi-isomorphism.

\begin{example}
    Every differentially graded algebra with underlying vector space \(V\) is
    naturally an \(A_\infty\) algebra by setting \(m_1\) to be the differential,
    \(m_2\) to be the product, and all higher multiplications to be \(0\). The
    cohomology complex \(H(V)\) has a differential \(m_1=0\) and inherits and
    \(m_2\) from \(V\), but the natural map \(f_1: H(V) \rightarrow V\) does not
    necessarily commute with the multiplication. For \(a,b\in H(V)\), we however
    do have that \(m_2(f_1(a),f_1(b))-f_1(m_2(a,b))\) is the boundary of some
    \(f_2(a,b)\), and this choice can be made for all \(a,b\) in a way that
    \(f_2\) is bilinear. Thus we have \(f_2: H(V)^{\otimes 2}\rightarrow V\)
    satisfying 
    \[ f_1(m_2(a,b)) =  m_2(f_1(a),f_1(b))+ m_1f_2(a,b). \]
    This is precisely the relation \eqref{Ainftymorphism} for \(n=2\). Now if we
    were to construct \(f_3:H(V)^{\otimes 3}\rightarrow V\) we would have to
    consider 
    \[f_2(m_2(a,b),c) + f_2(a,m_2(b,c)) + m_2(f_1(a), f_2(b,c)) +
    m_2(f_2(a,b), f_1(c)).\]
    Let \(m_3(a,b,c)\in H(V)\) be the cohomology class of this quantity, then it
    is equal to  \[m_1f_3(a,b,c) + f_1m_3(a,b,c)\] for some
    \(f_3(a,b,c)\in V\). Kadeishvili showed these and higher \(m_n\)s, \(f_n\)s
    can be inductively chosen to be multilinear maps on \(H(V)\) that satisfy
    relations \eqref{Ainftymult} and \eqref{Ainftymorphism}. This process is
    called \textit{homological perturbation}.
\end{example} 

\begin{theorem}[Kadeishvili] 
    Let \(V\) be a differentially graded algebra. Then its cohomology \(H(V)\) has
    an \(A_\infty\) structure with \(m_1=0\), \(m_2\) inherited from \(V\) such
    that there is a quasi-isomorphism \(H(V)\rightarrow V\) of \(A_\infty\)
    algebras lifting the identity of \(H(V)\). Moreover, this structure is
    unique up to isomorphism of \(A_\infty\) algebras. 
\end{theorem}

In fact, the converse is true as well: any \(A_\infty\) algebra with \(m_1=0\)
can be realised as the cohomology complex of some DGA. Thus dealing with
\(A_\infty\)-algebras is not too different from dealing with DGAs. 

\subsubsection{Constructing the deformation functor}
We can associate a deformation functor to an \(A_\infty\)-algebra \((V,m_i)\) by
considering zeroes of the \textit{Homotopy Maurer-Cartan function}
\begin{gather*}
    HMC: V_1\rightarrow V_2 \\
    HMC(a)=\sum_{n\geq 1} m_n(a^{\otimes n})
\end{gather*}
up to Gauge equivalence. Explicitly, we set 
\[\Def_{(V,m_i)}(R)=\left\{a\in V_1\otimes \mathfrak{m}_R\;|\; HMC(a) =
0\right\}\;/\;\sim\]
where the equivalence relation \(\sim\) is generated by an action of
\(V_0\otimes \mathfrak{m}_R\) on \(V_1\otimes \mathfrak{m}_R\). If \(V\) is a
DGA (i.e.\ \(m_n=0\) for \(n\geq 3\)) then this definition coincides with that
of \Cref{subsec-defbydga} (where the Lie bracket is given by the commutator).
Moreover, the functor only depends on the \(A_\infty\)-algebra up to
quasi-isomorphism.

We motivated the study of \(A_\infty\)-deformations by suggesting that a complex
with finite dimensional graded parts is easier to handle. To exhibit a concrete
benefit of this approach we will now show that whenever \((V,m_i)\) is an
\(A_\infty\)-algebra with \(\text{dim}V_1<\infty\), the corresponding
deformation functor is essentially prorepresentable by a power-series ring in
\(\text{dim}V_1\)-many variables. To begin, observe the \(A_\infty\)-structure gives us a map 
\[ m =  \bigoplus m_i : TV_1 \rightarrow V_2 \]
where \(TV_1 = \bigoplus_{n\geq 0} V_1^{\otimes n}\) is the tensor algebra on
\(V_1\).  Noting that \((TV_1)\dual \cong \widehat{T}(V_1\dual) =
\prod_{n\geq 0} (V_1\dual)^{\otimes n}\) is the completed tensor algebra on
\(V_1\dual\), we see that dualising \(m\) gives a map
\[m\dual : V_2\dual \rightarrow \widehat{T}(V_1\dual).\]
Write \((V_2\dual)\) for the ideal of \(\widehat{T}(V_1\dual)\) generated by the
image of \(m\dual\). The algebra \(\widehat{T}(V_1\dual)\) is naturally
augmented by the maximal ideal \(\prod_{n\geq 1}(V_1\dual)^{\otimes n}\), and
quotienting by \((V_2\dual)\) preserves this augmentation since the image of
\(HMC\,\dual\) has no degree \(0\) component. We have the following result.

\begin{lemma}\label{lemma-hmc-representable}
    If \(V\) is an \(A_\infty\)-algebra such that \(V_1\) is
    finite dimensional, then for any \(R\in \textnormal{Art}^1_k\) there is a
    functorial isomorphism 
    \[\Hom_{\textnormal{Alg}^1_k} \left(\frac{\widehat{T}(V_1\dual)}{(V_2\dual)},
    R\right) = \{v\in V_1\otimes \mathfrak{m}_R\;|\; HMC(v)=0\} .\] 
    \begin{proof} 
        Elements of \(V_1\otimes \mathfrak{m}_R\cong \Hom_k(V_1\dual,
        \mathfrak{m}_R)\) correspond precisely to morphisms of \(k\)-algebras
        \(\widehat{T}(V_1\dual)\rightarrow R\) that respect the augmentation
        (here we use that \(V_1\) is finite dimensional). Explicitly, \(v\in
        V_1\otimes \mathfrak{m}_R\) corresponds to the map 
        \begin{align*}
            f_v: \xi&\mapsto \xi\left(\sum_{i\geq 0}v^{\otimes i}\right)
        \end{align*}
        where we use the identification \(\widehat{T}(V_1\dual)\otimes
        TV_1\otimes R \cong R\), noting that the sum \(\sum_{i\geq 0} v^{\otimes i} \in
        TV_1\otimes R\) is finite because \(\mathfrak{m}_R\) is nilpotent. Now
        for a generator \(\xi=m\dual(w)\) of the ideal \((V_2\dual)\), we have 
        \begin{align*}
            f_v(m\dual(w)) = w\circ m \left(\sum_{i\geq 0}v^{\otimes i}\right)
            \;
                           = w(HMC(v))
        \end{align*}
        so \(f_v:\widehat{T}(V_1\dual)\) vanishes on the ideal \((V_2\dual)\) if
        and only if \(HMC(v)=0\). Thus we have the required isomorphism
        \[\Hom_{\text{Alg}^1_k} \left( \frac{\widehat{T}(V_1\dual)}{(V_2\dual)},
            R\right) = \{v\in V_1\otimes \mathfrak{m}_R\;|\; HMC(v)=0\}.\] 
        Functoriality follows from the fact that all isomorphisms involved were
        natural.
        \end{proof}
\end{lemma}

\subsubsection{The \(A_\infty\)-deformation theory of a point}
Going back to the deformation problem of a module \(M\) over an associative
algebra \(A\), we see that the complex \(\Ext_A(M,M)\) has an
\(A_\infty\)-structure that makes it quasi-isomorphic to the differentially
graded endomorphism algebra \(\text{End}_A(P_\bullet,P_\bullet)\) where
\(P_\bullet\rightarrow M\) is a projective resolution. Moreover, this
\(A_\infty\)-algebra controls the deformation functor \(\Def_M\) in the sense
described above.

The statement is cleanest when \(M\) is simple. To use
\Cref{lemma-hmc-representable}, we need to place an additional constraint on
\(M\), namely we require that \(\Ext^1_A(M,M)\) is finite dimensional (this
holds automatically when \(A\) is Noetherian).

\begin{theorem}[Segal]\label{thm-segal}
    If \(M\) is a simple \(A\)-module such that \(\Ext^1_A(M,M)\) finite
    dimensional, then the functor \(\Def_M: \textnormal{Art}^1_k \rightarrow
    \text{Set}\) is given by 
    \[
        \Def_M(R) = \Hom_{\textnormal{Alg}^1_k}
        \left(\frac{\widehat{T}(\Ext^1_A(M,M)\dual)}{(\Ext_A^2(M,M))\dual},
        R\right)\;/\;\{\text{Inner automorphisms of \(R\)}\} 
    \]
    \begin{proof}
        We know that \(\Def_M=\Def_{\Ext_A(M,M)}\) is given by zeroes of the
        homotopy Maurer-Cartan function up to Gauge equivalence. From
        \Cref{lemma-hmc-representable} we have that the zeros of the homotopy
        Maurer-Cartan function for \(\Ext_A(M,M)\otimes R\) are precisely given
        by 
        \[ 
            \Hom_{\text{Alg}^1_k} \left(
            \frac{\widehat{T}(\Ext^1_A(M,M)\dual)}{(\Ext_A^2(M,M))\dual}, R\right) 
        \]
        so it suffices to show that the gauge action is precisely by inner
        automorphisms of \(R\). 

        Since \(M\) is simple, we have \(\Ext^0_A(M,M)\cong k\) so that the
        Gauge group has underlying set \(\mathfrak{m}_R\). The element \(r\in
        \mathfrak{m}_R\) acts on
        \(f:\frac{\widehat{T}(\Ext^1_A(M,M)\dual)}{(\Ext_A^2(M,M))\dual}\rightarrow
        R\) by mapping it to \(r\ast f: a\mapsto (1+r)^{-1}a(1+r)\). Since
        conjugation by \(u+r\) for \(u\in R^\times\) a unit is the same as
        conjugation by \(1+u^{-1}r\), it follows that the gauge action
        identifies morphisms that differ by an inner automorphism of \(R\).
    \end{proof}
\end{theorem}

When \(M\) is one-dimensional, i.e.\ given by a surjection \(\mu:A\rightarrow
k\) with kernel \(p\), we can use this to obtain a presentation of the
completion \(\widehat{A}_p\).

\begin{corollary}
    Let \(\widehat{A}_p\) be the completion of \(A\) at a maximal ideal \(p\).
    Writing \(M=A/p\), if \(\Ext^1_A(M,M)\) is finite dimensional then we have
    the isomorphism 
    \[\widehat{A}_p =
    \frac{\widehat{T}(\Ext^1_A(M,M)\dual)}{(\Ext_A^2(M,M)\dual)}.\]
    \begin{proof}[Proof (sketch)] 
        Forgetting the action of inner automorphisms, we saw in
        \Cref{lemma-prorep-by-completion} that the functor
        \(\Def_M\) is essentially prorepresented by \(\widehat{A}_p\). Thus by
        uniqueness of the prorepresenting object (i.e.\ Yoneda's lemma), we
        expect the result to hold and all that remains to be checked is that the
        uniqueness argument holds with little change even after we quotient by
        inner automorphisms.
    \end{proof}
\end{corollary}

\begin{remark}
    In \cite{segal__2008}, Ed Segal proves the results of this section in the
    slightly more general setting of \textit{\(n\)-pointed Algebras}, which are
    algebras equipped with augmentation maps to \(k^n\). There is a natural
    test category of n-pointed Artinian (semilocal) rings, and the deformation
    problem can be interpreted as deforming a \textit{set of points}. It is here
    that the noncommutativity of test objects shines-- the prorepresenting
    algebra can be realised as a path algebra on an \(n\)-pointed quiver, but
    passing to the abelianisation kills all paths between distinct vertices thus
    decomposing the deformation problem for \(n\) points into \(n\) distinct
    deformation problems for single points.

    The proofs are essentially identical to the one-pointed case presented here,
    except with more notational baggage to keep track of the additional
    structure.
\end{remark}
